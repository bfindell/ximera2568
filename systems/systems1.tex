\documentclass{ximera}

\input{../preamble}



\title{Systems Again}
\author{\phantom{Dr. Golubitsky}}
\date{Due: TBA}

%\makeatletter
%\newlabel{c1.4.1a}{{5}{13}}
%\newlabel{c1.4.1c}{{8}{13}}
%\newlabel{c2.3.6a}{{1}{34}}
%\newlabel{c2.3.6c}{{3}{34}}
%\newlabel{c2.3.7a}{{4}{34}}
%\newlabel{c2.3.7c}{{6}{34}}
%\newlabel{e:refexamp6}{{2.3.14}{34}}
%\newlabel{c2.3.10a}{{10}{35}}
%\newlabel{c2.3.10b}{{11}{35}}
%\newlabel{c2.3.11a}{{12}{35}}
%\newlabel{c2.3.11d}{{15}{35}}
%\newlabel{c2.3.1a}{{22}{36}}
%\newlabel{c2.3.1c}{{24}{36}}
%\newlabel{number}{{2.4.6}{40}}
%\newlabel{c2.4.3a}{{5}{42}}
%\newlabel{c2.4.3d}{{8}{42}}
%\newlabel{c2.4.4a}{{9}{42}}
%\newlabel{c2.4.4c}{{11}{43}}
%\newlabel{c2.5.2a}{{4}{48}}
%\newlabel{c2.5.2c}{{6}{48}}
%\makeatother

\begin{document}
\begin{abstract}
Systems  
\end{abstract}
\maketitle


\problemlabel

\exerciselabel{5}{2.2}\begin{exercise} \label{c2.2.9}
\begin{itemize}
\item[(a)] Find a vector $u$ normal to the plane $2x+2y+z=3$.
  \begin{prompt}
    \begin{validator}[(ux==uy) && (uy==2*uz) && (uy != 0)]
      \(
        u = \left( \answer[format=integer,id=ux]{2}, \answer[format=integer,id=uy]{2}, \answer[format=integer,id=uz]{1} \right)
      \)
    \end{validator}
  \end{prompt}
\item[(b)] Find a vector $v$ normal to the plane $x+y+2z=4$.
  \begin{prompt}
    \begin{validator}[(vx==vy) && (2*vy==vz) && (vy != 0)]
      \(
        v = \left( \answer[format=integer,id=vx]{1}, \answer[format=integer,id=vy]{1}, \answer[format=integer,id=vz]{2} \right)
      \)
    \end{validator}
  \end{prompt}
\item[(c)] Find the cosine of the angle $\theta$ between the vectors $u$ and $v$.
  \begin{prompt}
    \[
      \cos \theta = \answer{\frac{2}{\sqrt{6}}}
    \]
  \end{prompt}
\end{itemize}

\begin{solution}

(a) $u = (2,2,1)$, since we know that the normal vector to the plane
$ax + by + cz = d$ is $(a,b,c)$.

(b) $v = (1,1,2)$.

(c) $\cos\theta = \frac{u \cdot v}{||u||\;||v||} = \frac{2}{\sqrt{6}}$.
In \Matlabp, type {\tt acos(2/sqrt(6))*180/pi} to obtain $\theta =
35.2644^\circ$.

\end{solution}
\end{exercise}

\problemlabel

\noindent Determine whether the given matrix is in reduced echelon form.

\exerciselabel{1}{2.3}\begin{exercise} \label{c2.3.6a}
$\left(\begin{array}{rrrr}
1 & -1 &  0 &   1   \\
0 &  1 &  0 &  -6    \\
         0 &  0 &  1 &   0   \end{array}\right)$.
     \begin{multipleChoice}
       \choice{The matrix is in reduced echelon form.}         
       \choice[correct]{The matrix is not in reduced echelon form.}         
     \end{multipleChoice}

\begin{feedback}
The matrix is in (row) echelon form.  But it is not in \textbf{reduced} echelon form because of the $-1$ above the pivot in row 2. 

\end{feedback}
\end{exercise}



\problemlabel

\noindent Consider the augmented matrices representing systems of linear equations, and decide \begin{itemize} \item[(a)] if there are zero, one or infinitely many solutions, and \item[(b)] if solutions are not unique, how many variables can be assigned arbitrary values. \end{itemize}

\exerciselabel{14}{2.3}\begin{exercise} \label{c2.3.11c}
$\left(\begin{array}{ccc|c}  1 & 0 & 2 & 1\\ 0 & 5 & 0 & 2 \\ 0 & 0 & 4 & 3
       \end{array}\right)$.
     \begin{multipleChoice}
       \choice{There are no solutions.}
       \choice[correct]{There is exactly one solution.}
       \choice{There are infinitely many solutions.}
     \end{multipleChoice}
     \begin{hint}
       The row-reduced form of the matrix is:
\[
\left(\begin{array}{rrr|r} 1 & 0 & 2 & 1 \\ 0 & 1 & 0 & \frac{2}{5}
\\ 0 & 0 & 1 & \frac{3}{4}\end{array}\right).
\]
     \end{hint}

\begin{solution}

\ans The system has a unique solution.

\soln The row-reduced form of the matrix is:
\[
\left(\begin{array}{rrr|r} 1 & 0 & 2 & 1 \\ 0 & 1 & 0 & \frac{2}{5}
\\ 0 & 0 & 1 & \frac{3}{4}\end{array}\right).
\]

\end{solution}
\end{exercise}

\matlabproblemlabel

\noindent Use elementary row operations and \Matlab to put each of the given matrices into row echelon form.  Suppose that the matrix is the augmented matrix for a system of linear equations.  Is the system consistent or inconsistent?

\exerciselabel{24}{2.3}\begin{computerExercise} \label{c2.3.1c}
\[
\left(\begin{array}{rrrr}
 -2 & 1 &  9 & 1\\
  3 & 3 & -4 & 2\\
  1 & 4 &  5 & 5
\end{array}\right).
\]

\begin{solution}
The row-reduced matrix is:
\begin{verbatim}
A =
    1.0000   -0.5000   -4.5000   -0.5000
         0    1.0000    2.1111    0.7778
         0         0         0    2.0000
\end{verbatim}
This matrix represents an inconsistent linear system.


\end{solution}
\end{computerExercise}

\problemlabel

\exerciselabel{3}{2.4}\begin{exercise} \label{c2.4.2}
The augmented matrix of a consistent system of five equations in seven
unknowns has rank equal to three.  How many parameters are needed to
specify all solutions?
\begin{prompt}
  There are $\answer{4}$ parameters needed to specify all solutions.
\end{prompt}
\begin{hint}
  According to Theorem 2.4.6, $n - \ell$
parameters are needed to parameterize the set of all solutions of a
linear system, where $n$ is the number of unknowns, and $\ell$ is the
rank of the reduced echelon matrix.  In this case, $n = \answer{7}$ and $\ell =
\answer{3}$.
\end{hint}

\end{exercise}

\matlabproblemlabel

\noindent Use {\tt rref} on the given augmented matrices to determine whether the associated system of linear equations is consistent or inconsistent.  If the equations are consistent, then determine how many parameters are needed to enumerate all solutions.

\exerciselabel{5}{2.4}\begin{computerExercise} \label{c2.4.3a}
\begin{matlabEquation}\label{MATLAB:17}
A = \left(\begin{array}{rrrrr|r}
2 & 1 & 3 & -2 & 4 & 1 \\
5 & 12 & -1 & 3 & 5 & 1 \\
-4  &  -21 &    11  &  -12  &    2  &    1  \\
23  &  59  &  -8   & 17  &  21  &   4
\end{array}\right) \quad
\end{matlabEquation}

\ans Matrix $A$ is consistent and requires 3 parameters to enumerate
all solutions.

\soln
\begin{verbatim}
rref(A) = 
    1.0000         0    1.9474   -1.4211    2.2632    0.5789
         0    1.0000   -0.8947    0.8421   -0.5263   -0.1579
         0         0         0         0         0         0
         0         0         0         0         0         0
\end{verbatim}

\end{computerExercise}

\matlabproblemlabel

\noindent Compute the rank of the given matrix.

\exerciselabel{10}{2.4}\begin{computerExercise} \label{c2.4.4b}
$\left(\begin{array}{rrrr} 2 & 1 & 0 & 1\\
	-1 & 3 & 2 & 4\\ 5 & -1 & 2 & -2\end{array}\right)$.

\begin{solution}
The rank of the matrix is $3$.

\end{solution}
\end{computerExercise}


\end{document}
