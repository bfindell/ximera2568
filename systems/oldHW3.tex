\documentclass{ximera}

\usepackage{pgf,tikz}
\usepackage{mathrsfs}
\usetikzlibrary{shapes,arrows}
\usepackage{framed}
\pgfplotsset{compat=1.13}

\graphicspath{
  {./}
  {ximeraTutorial/}
}

\newenvironment{sectionOutcomes}{}{}


% Below is the preamble from the textbook in /laode, by Golubitsky and Dellnitz
% ... with problematic stuff commented out
% Followed by the hw-preamble from worksheet-builder

%\usepackage{ulem}
\usepackage[normalem]{ulem}

\epstopdfsetup{outdir=./}

\usepackage{morewrites}
\makeatletter
\newcommand\subfile[1]{%
\renewcommand{\input}[1]{}%
\begingroup\skip@preamble\otherinput{#1}\endgroup\par\vspace{\topsep}
\let\input\otherinput}
\makeatother

\newcommand{\EXER}{}
\newcommand{\includeexercises}{\EXER\directlua{dofile(kpse.find_file("exercises","lua"))}}

\newenvironment{computerExercise}{\begin{exercise}}{\end{exercise}}

%\newcounter{ccounter}
%\setcounter{ccounter}{1}
%\newcommand{\Chapter}[1]{\setcounter{chapter}{\arabic{ccounter}}\chapter{#1}\addtocounter{ccounter}{1}}

%\newcommand{\section}[1]{\section{#1}\setcounter{thm}{0}\setcounter{equation}{0}}

%\renewcommand{\theequation}{\arabic{chapter}.\arabic{section}.\arabic{equation}}
%\renewcommand{\thefigure}{\arabic{chapter}.\arabic{figure}}
%\renewcommand{\thetable}{\arabic{chapter}.\arabic{table}}

%\newcommand{\Sec}[2]{\section{#1}\markright{\arabic{ccounter}.\arabic{section}.#2}\setcounter{equation}{0}\setcounter{thm}{0}\setcounter{figure}{0}}
  
\newcommand{\Sec}[2]{\section{#1}}

\setcounter{secnumdepth}{2}
%\setcounter{secnumdepth}{1} 

%\newcounter{THM}
%\renewcommand{\theTHM}{\arabic{chapter}.\arabic{section}}

\newcommand{\trademark}{{R\!\!\!\!\!\bigcirc}}
%\newtheorem{exercise}{}

\newcommand{\dfield}{{\sf SlopeField}}

\newcommand{\pplane}{{\sf PhasePlane}}

\newcommand{\PPLANE}{{\sf PHASEPLANE}}

% BADBAD: \newcommand{\Bbb}{\bf}. % Package amsfonts Warning: Obsolete command \Bbb; \mathbb should be used instead.

\newcommand{\R}{\mbox{$\mathbb{R}$}}
\let\C\relax
\newcommand{\C}{\mbox{$\mathbb{C}$}}
\newcommand{\Z}{\mbox{$\mathbb{Z}$}}
\newcommand{\N}{\mbox{$\mathbb{N}$}}
\newcommand{\D}{\mbox{{\bf D}}}

\newcommand{\WW}{\mathcal{W}}

\usepackage{amssymb}
%\newcommand{\qed}{\hfill\mbox{\raggedright$\square$} \vspace{1ex}}
%\newcommand{\proof}{\noindent {\bf Proof:} \hspace{0.1in}}

\newcommand{\setmin}{\;\mbox{--}\;}
\newcommand{\Matlab}{{M\small{AT\-LAB}} }
\newcommand{\Matlabp}{{M\small{AT\-LAB}}}
\newcommand{\computer}{\Matlab Instructions}
\renewcommand{\computer}{M\small{ATLAB} Instructions}
\newcommand{\half}{\mbox{$\frac{1}{2}$}}
\newcommand{\compose}{\raisebox{.15ex}{\mbox{{\scriptsize$\circ$}}}}
\newcommand{\AND}{\quad\mbox{and}\quad}
\newcommand{\vect}[2]{\left(\begin{array}{c} #1_1 \\ \vdots \\
 #1_{#2}\end{array}\right)}
\newcommand{\mattwo}[4]{\left(\begin{array}{rr} #1 & #2\\ #3
&#4\end{array}\right)}
\newcommand{\mattwoc}[4]{\left(\begin{array}{cc} #1 & #2\\ #3
&#4\end{array}\right)}
\newcommand{\vectwo}[2]{\left(\begin{array}{r} #1 \\ #2\end{array}\right)}
\newcommand{\vectwoc}[2]{\left(\begin{array}{c} #1 \\ #2\end{array}\right)}

\newcommand{\ignore}[1]{}


\newcommand{\inv}{^{-1}}
\newcommand{\CC}{{\cal C}}
\newcommand{\CCone}{\CC^1}
\newcommand{\Span}{{\rm span}}
\newcommand{\rank}{{\rm rank}}
\newcommand{\trace}{{\rm tr}}
\newcommand{\RE}{{\rm Re}}
\newcommand{\IM}{{\rm Im}}
\newcommand{\nulls}{{\rm null\;space}}

\newcommand{\dps}{\displaystyle}
\newcommand{\arraystart}{\renewcommand{\arraystretch}{1.8}}
\newcommand{\arrayfinish}{\renewcommand{\arraystretch}{1.2}}
\newcommand{\Start}[1]{\vspace{0.08in}\noindent {\bf Section~\ref{#1}}}
\newcommand{\exer}[1]{\noindent {\bf \ref{#1}}}
\newcommand{\ans}{\textbf{Answer:} }
\newcommand{\matthree}[9]{\left(\begin{array}{rrr} #1 & #2 & #3 \\ #4 & #5 & #6
\\ #7 & #8 & #9\end{array}\right)}
\newcommand{\cvectwo}[2]{\left(\begin{array}{c} #1 \\ #2\end{array}\right)}
\newcommand{\cmatthree}[9]{\left(\begin{array}{ccc} #1 & #2 & #3 \\ #4 & #5 &
#6 \\ #7 & #8 & #9\end{array}\right)}
\newcommand{\vecthree}[3]{\left(\begin{array}{r} #1 \\ #2 \\
#3\end{array}\right)}
\newcommand{\cvecthree}[3]{\left(\begin{array}{c} #1 \\ #2 \\
#3\end{array}\right)}
\newcommand{\cmattwo}[4]{\left(\begin{array}{cc} #1 & #2\\ #3
&#4\end{array}\right)}

\newcommand{\Matrix}[1]{\ensuremath{\left(\begin{array}{rrrrrrrrrrrrrrrrrr} #1 \end{array}\right)}}

\newcommand{\Matrixc}[1]{\ensuremath{\left(\begin{array}{cccccccccccc} #1 \end{array}\right)}}



\renewcommand{\labelenumi}{\theenumi}
\newenvironment{enumeratea}%
{\begingroup
 \renewcommand{\theenumi}{\alph{enumi}}
 \renewcommand{\labelenumi}{(\theenumi)}
 \begin{enumerate}}
 {\end{enumerate}
 \endgroup}

\newcounter{help}
\renewcommand{\thehelp}{\thesection.\arabic{equation}}

%\newenvironment{equation*}%
%{\renewcommand\endequation{\eqno (\theequation)* $$}%
%   \begin{equation}}%
%   {\end{equation}\renewcommand\endequation{\eqno \@eqnnum
%$$\global\@ignoretrue}}

\author{Martin Golubitsky and Michael Dellnitz}

%\newenvironment{matlabEquation}%
%{\renewcommand\endequation{\eqno (\theequation*) $$}%
%   \begin{equation}}%
%   {\end{equation}\renewcommand\endequation{\eqno \@eqnnum
% $$\global\@ignoretrue}}

\newcommand{\soln}{\textbf{Solution:} }
\newcommand{\exercap}[1]{\centerline{Figure~\ref{#1}}}
\newcommand{\exercaptwo}[1]{\centerline{Figure~\ref{#1}a\hspace{2.1in}
Figure~\ref{#1}b}}
\newcommand{\exercapthree}[1]{\centerline{Figure~\ref{#1}a\hspace{1.2in}
Figure~\ref{#1}b\hspace{1.2in}Figure~\ref{#1}c}}
\newcommand{\para}{\hspace{0.4in}}

%\usepackage{ifluatex}
%\ifluatex
%\ifcsname displaysolutions\endcsname%
%\else
%\renewenvironment{solution}{\suppress}{\endsuppress}
%\fi
%\else
%\renewenvironment{solution}{}{}
%\fi
%
%\ifcsname answer\endcsname
%\renewcommand{\answer}{}
%\fi

%\ifxake
%\newenvironment{matlabEquation}{\begin{equation}}{\end{equation}}
%\else
\newenvironment{matlabEquation}%
{\let\oldtheequation\theequation\renewcommand{\theequation}{\oldtheequation*}\begin{equation}}%
  {\end{equation}\let\theequation\oldtheequation}
%\fi

%\makeatother

\newcommand{\RED}[1]{{\color{red}{#1}}} 


%%
%%
%% Worksheet-builder preamble
%%
%%

\usepackage{xcolor}
\renewenvironment{solution}{\color{blue}}{\color{black}}
\renewenvironment{computerExercise}{\begin{exercise}\textsc{(matlab)} }{\end{exercise}}

%\usepackage{environ}
%\RenewEnviron{prompt}{}
%\RenewEnviron{hint}{}
%\RenewEnviron{multipleChoice}{}
%\RenewEnviron{feedback}{}

\renewcommand{\ans}{\noindent\textbf{Answer: }}
\renewcommand{\soln}{\noindent\textbf{Solution: }}

%\renewcommand{\answer}[2][]{#2}

% if you want to hide solutions, uncomment the following
%\usepackage{comment}\excludecomment{solution}

\def\isitmatlab{}
\newcommand{\matlab}{\def\isitmatlab{ (MATLAB)}}

\makeatletter
\newcommand{\exerciselabel}[2]{\textbf{\textsection #2, Exercise #1\isitmatlab.}\def\@currentlabel{#1}\def\isitmatlab{}}
\makeatother

\newcounter{problemx}
\newcommand{\problemlabel}{\refstepcounter{problemx}\section*{Problem \arabic{problemx}}}
\newcommand{\matlabproblemlabel}{\refstepcounter{problemx}\section*{Problem \arabic{problemx} (MATLAB)}}
%\newcommand{\problemlabel}{\refstepcounter{problem}\section*{Problem}}
%\newcommand{\matlabproblemlabel}{\refstepcounter{problem}\section*{Problem (MATLAB)}}



% if you want to hide solutions, uncomment the following
%\usepackage{comment}\excludecomment{solution}

\title{Math 2568 Old Homework 3}
\author{\phantom{Dr. Golubitsky}}
\date{Due: TBA}

%\makeatletter
%\newlabel{c2.4.1}{{1}{83}}
%\newlabel{c2.4.1b}{{2}{84}}
%\newlabel{chap:prelim}{{1}{2}}
%\newlabel{eq:avect}{{3.1.6}{108}}
%\makeatother
\begin{document}
\begin{abstract}
Old Homework 3. 
\end{abstract}
\maketitle


\problemlabel

\noindent Row reduce the given  matrix to reduced echelon form by hand and determine its rank.

\exerciselabel{1}{2.4}\begin{exercise} \label{c2.4.1}
$A=\left(\begin{array}{rrrr}
1 &  2 & 1 & 6\\
3 &  6 & 1 & 14\\
1 &  2 & 2 & 8
         \end{array}\right)$
       \begin{prompt}
       The reduced echelon form of the matrix is:
\[
A = \left(\begin{array}{rrrr} 1 & \answer{2} & \answer{0} & \answer{4} \\ 0 & \answer{0} & \answer{1} & \answer{2} \\ 0 & 0 & \answer{0}
& 0\end{array}\right)
\]
The rank of $A$ is $\answer{2}$, since the reduced echelon matrix has $\answer{2}$ nonzero
rows.         
       \end{prompt}

\begin{solution}

The reduced echelon form of the matrix is:
\[
A = \left(\begin{array}{rrrr} 1 & 2 & 0 & 4 \\ 0 & 0 & 1 & 2 \\ 0 & 0 & 0
& 0\end{array}\right)
\]
The rank of $A$ is two, since the reduced echelon matrix has two nonzero
rows.

\end{solution}
\end{exercise}

\problemlabel



\exerciselabel{3}{2.4}\begin{exercise}\label{C2S4_1c}

How many solutions does the equation
\[
A \begin{pmatrix}x_1 \\ x_2 \\ x_3\end{pmatrix} = 
 \begin{pmatrix}2 \\ 1 \\ 2\end{pmatrix}
\]
have for the following choices of $A$.  Explain your reasoning.
\begin{enumeratea}
\item $A = \begin{pmatrix}1 & 0 & 1\\ 0 & 1 & 0 \\ 0 & 0 & 0\end{pmatrix}$

\item $A = \begin{pmatrix}1 & 3 & 1\\ 2 & 1 & 0 \\ 0 & 0 & 1\end{pmatrix}$

\item $A = \begin{pmatrix}1 & 1 & 1\\ 1 & 2 & 1 \\ 1 & 1 & 1\end{pmatrix}$
\end{enumeratea}


  
\begin{solution}

\ans (a) no solutions; (b) 1 solution; (c) infinitely many solutions

\soln 
\begin{enumeratea}
\item The third equation in this system is $0 = 1$ and that is inconsistent.

\item $A$ is invertible; so there is 1 solution

\item Reduce the augmented matrix to echelon form. The rank of $A$ is $2$ as is the rank of the augmented matrix.  Therefore, there exists a one-parameter set of solutions.
\end{enumeratea}

\end{solution}
\end{exercise}

\problemlabel

% To which section does your exercise belong? 

\exerciselabel{6}{2.4}\begin{exercise}\label{c2.4.2b.2}

Consider the system of equations
\[
\begin{array}{rcl}
x_1 + 3x_3 & = & 1 \\
-x_1+2x_2-3x_3 & = & 1\\
2x_2 + ax_3 & = & b
\end{array}
\]
For which real numbers $a$ and $b$ does the system have no solutions, a unique solution, or infinitely many solutions?  Your answer should subdivide the $ab$-plane into three disjoint sets.
  
\begin{solution}

\ans Unique solutions occur when $a\neq 0$; no solution occurs when $a=0$ and $b\neq 2$; and infinitely many solutions exist when $a = 0$ and $b =2$.

\soln 
Use row reduction on the augmented matrix to obtain
\[
\Matrix{1 & 0 & 3 & 1\\ -1 & 2 & -3 & 1 \\ 0 & 2 & a & b} \to
\Matrix{1 & 0 & 3 & 1\\  0 & 2 & 0 & 2 \\ 0 & 2 & a & b} \to
\]
\[
\Matrix{1 & 0 & 3 & 1\\  0 & 2 & 0 & 2 \\ 0 & 0 & a & b - 2} \to
\Matrix{1 & 0 & 3 & 1\\  0 & 1 & 0 & 1 \\ 0 & 0 & a & b - 2} 
\]
If $a\neq 0$ the system has a unique solution. If $a = 0$ we obtain the echelon form matrix
\[
\Matrix{1 & 0 & 3 & 1\\  0 & 1 & 0 & 1 \\ 0 & 0 & 0 & b - 2}
\]  
There are no solutions if $b\neq 2$ and infinitely many solutions if $b = 2$.

\end{solution}
\end{exercise}

\problemlabel

\exerciselabel{14}{2.4}\begin{exercise} \label{A:2.4.1}
  Prove that the rank of an $m \times n$ matrix $A$ is less than or equal to
  the minimum of $m$ and $n$.

\begin{solution}
Suppose $A$ is row equivalent to the $m \times n$ reduced row echelon matrix $E$.  The rank of $A$ equals the number of pivots in $E$. Since there is at most $1$ pivot in each column, the number of pivots is less than or equal to the number of columns $n$ of $E$.  Similarly, since each row of $E$ contains at most one pivot, the number of pivots in $E$ is at most the number $m$ of rows of $E$.  It follows that the rank of $A$ is less than or equal to both $m$ and $n$ and hence the minimum of $m$ and $n$.  
\end{solution}
\end{exercise}

\problemlabel

\exerciselabel{1}{3.1}\begin{exercise} \label{c4.1.1}
Let
\[
A = \begin{pmatrix}2 & 1 \\ -1 & 4\end{pmatrix}\qquad\textrm{and}\qquad x = \begin{pmatrix}3 \\ -2\end{pmatrix}.
\]
Compute $Ax\begin{prompt}=\left(\begin{array}{r} \answer{4} \\ \answer{-11}\end{array}\right)\end{prompt}$.
\begin{hint}
  \[
Ax =
\left(\begin{array}{rr} 2 & 1 \\ -1 & 4\end{array}\right)
\left(\begin{array}{r} 3 \\ -2\end{array}\right) =
\left(\begin{array}{r} 6 - 2 \\ -3 - 8\end{array}\right) =
\left(\begin{array}{r} 4 \\ -11\end{array}\right)
\]
\end{hint}

\begin{solution}

\[
Ax =
\left(\begin{array}{rr} 2 & 1 \\ -1 & 4\end{array}\right)
\left(\begin{array}{r} 3 \\ -2\end{array}\right) =
\left(\begin{array}{r} 6 - 2 \\ -3 - 8\end{array}\right) =
\left(\begin{array}{r} 4 \\ -11\end{array}\right)
\]

\end{solution}
\end{exercise}

\problemlabel

\exerciselabel{7}{3.1}\begin{exercise} \label{c4.1.b3}
Let
\[
A=\left(
\begin{array}{rrrr}
 a_{11} & a_{12} & \cdots & a_{1n} \\
 a_{21} & a_{22} & \cdots & a_{2n}  \\
 \vdots & \vdots &        & \vdots  \\
 a_{m1} & a_{m2} & \cdots & a_{mn}
\end{array}
\right)\quad\mbox{and}\quad
x =
\left( \begin{array}{r} x_1\\ x_2\\ \vdots\\ x_n\end{array}\right).
\]
Denote the columns of the matrix $A$ by
\[
A_1 =
\left(\begin{array}{c} a_{11}\\ a_{21}\\ \vdots\\
a_{m1}\end{array}\right),\quad
A_2 =
\left(\begin{array}{c} a_{12}\\ a_{22}\\ \vdots\\
a_{m2}\end{array}\right),\quad
\cdots\quad
A_n =
\left(\begin{array}{c} a_{1n}\\ a_{2n}\\ \vdots\\ a_{mn}\end{array}\right).
\]
Show that the matrix vector product $Ax$ can be written as
\[
Ax = x_1 A_1 + x_2 A_2 + \cdots + x_n A_n,
\]
where $x_j A_j$ denotes scalar multiplication. %(see Chapter~\ref{chap:prelim}).

\begin{solution}

Compute $Ax$ directly:
\[ Ax = \left(\begin{array}{c} x_1a_{11} + x_2a_{12} + \cdots +
x_na_{1n} \\  x_1a_{21} + x_2a_{22} + \cdots + x_na_{2n} \\
\\ \vdots \\ x_1a_{m1} + x_2a_{m2} + \cdots + x_na_{mn}
\end{array}\right) = x_1\left(\begin{array}{r} a_{11} \\ a_{21} \\
\vdots \\ a_{m1} \end{array}\right) + x_2\left(\begin{array}{r}
a_{12} \\ a_{22} \\ \vdots \\ a_{m2} \end{array}\right) + \cdots
+ x_n\left(\begin{array}{r} a_{1n} \\ a_{2n} \\
\vdots \\ a_{mn} \end{array}\right). \]
So, it is indeed true that $Ax = x_1A_1 + x_2A_2 + \cdots
+ x_nA_n$.


\end{solution}
\end{exercise}

\problemlabel

\exerciselabel{9}{3.1}\begin{exercise} \label{c4.1.4}
Write the system of linear equations
\begin{eqnarray*}
2x_1 + 3x_2 - 2x_3 & = & 4\\
6x_1 -5x_3 & = & 1
\end{eqnarray*}
in the matrix form $Ax=b$.

\begin{solution}

\[
\left(\begin{array}{rrr} 2 & 3 & -2 \\ 6 & 0 & -5\end{array}\right) 
\left(\begin{array}{r} x_1 \\ x_2 \\ x_3\end{array}\right) = 
\left(\begin{array}{r} 4 \\ 1\end{array}\right)
\]


\end{solution}
\end{exercise}

\problemlabel

\exerciselabel{10}{3.1}\begin{exercise} \label{c4.1.6}
Find all solutions to
\[
\left(\begin{array}{rrrr} 1 & 3 & -1 & 4 \\ 2 & 1 & 5 & 7 \\
3 & 4 & 4 & 11 \end{array} \right)
\left(\begin{array}{c} x_1 \\ x_2 \\ x_3 \\ x_4\end{array}\right) =
\left(\begin{array}{c} 14 \\ 17 \\31 \end{array}\right).
\]

\begin{solution}

\ans All solutions are of the form
\[ \left(\begin{array}{r} x_1 \\ x_2 \\ x_3 \\ x_4\end{array}\right) =
\left(\begin{array}{c} \frac{37}{5} - \frac{16}{5}x_3 - \frac{17}{5}x_4 \\
\frac{11}{5} + \frac{7}{5}x_3 - \frac{1}{5}x_4 \\ x_3 \\ x_4\end{array}\right)
\]
where $x_3$ and $x_4$ are free parameters.

\soln Create the augmented matrix
\[ \left(\begin{array}{rrrr|r}
1 & 3 & -1 & 4 & 14 \\
2 & 1 & 5 & 7 & 17 \\
3 & 4 & 4 & 11 & 31\end{array}\right) \]
which can be row reduced to
\[ \left(\begin{array}{rrrr|r}
1 & 0 & \frac{16}{5} & \frac{17}{5} & \frac{37}{5} \\
0 & 1 & -\frac{7}{5} & \frac{1}{5} & \frac{11}{5} \\
0 & 0 & 0 & 0 & 0\end{array}\right), \]
yielding the desired solution.

\end{solution}
\end{exercise}

\problemlabel

\exerciselabel{13}{3.1}\begin{exercise} \label{c4.1.9}
Is there an upper triangular $2\times 2$ matrix $A$ such that
\begin{equation}  \label{eq:avect}
A\begin{pmatrix}1\\0\end{pmatrix} = \begin{pmatrix} 1 \\ 2\end{pmatrix}?
\end{equation}
Is there a symmetric $2\times 2$ matrix $A$ satisfying \eqref{eq:avect}?

\begin{solution}

\ans There is no $2 \times 2$ upper triangular matrix $A$ that
satisfies equation \eqref{eq:avect}, but any symmetric matrix $A$ of the form
\[ A = \mattwo{1}{2}{2}{a_{22}}, \]
where $a_{22}$ is a real number, satisfies \eqref{eq:avect}.

\soln Let $A$ be the upper triangular matrix
\[ \mattwo{a_{11}}{a_{12}}{0}{a_{22}}. \]
The resulting matrix equation
\[ \mattwo{a_{11}}{a_{12}}{0}{a_{22}}
\vectwo{1}{0} = \vectwo{1}{2} \]
yields the linear equations
\[ \begin{array}{rcl}
a_{11} & = & 1 \\
0 & = & 2.\end{array} \]
The second equation is inconsistent, so there is no solution.

\para Then let $A$ be the symmetric matrix
\[ \mattwo{a_{11}}{a_{12}}{a_{12}}{a_{22}}. \]
Write the matrix equation
\[ \mattwo{a_{11}}{a_{12}}{a_{12}}{a_{22}}
\vectwo{1}{0} = \vectwo{1}{2}, \]
from which we obtain the consistent linear system
\[ \begin{array}{rcl}
a_{11} & = & 1 \\
a_{12} & = & 2.\end{array} \]

\end{solution}
\end{exercise}

\end{document}
