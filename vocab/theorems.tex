\documentclass{ximera}
%\documentclass[space,handout,nooutcomes]{ximera}

%\usepackage{pgf,tikz}
\usepackage{mathrsfs}
\usetikzlibrary{shapes,arrows}
\usepackage{framed}
\pgfplotsset{compat=1.13}

\graphicspath{
  {./}
  {ximeraTutorial/}
}

\newenvironment{sectionOutcomes}{}{}


% Below is the preamble from the textbook in /laode, by Golubitsky and Dellnitz
% ... with problematic stuff commented out
% Followed by the hw-preamble from worksheet-builder

%\usepackage{ulem}
\usepackage[normalem]{ulem}

\epstopdfsetup{outdir=./}

\usepackage{morewrites}
\makeatletter
\newcommand\subfile[1]{%
\renewcommand{\input}[1]{}%
\begingroup\skip@preamble\otherinput{#1}\endgroup\par\vspace{\topsep}
\let\input\otherinput}
\makeatother

\newcommand{\EXER}{}
\newcommand{\includeexercises}{\EXER\directlua{dofile(kpse.find_file("exercises","lua"))}}

\newenvironment{computerExercise}{\begin{exercise}}{\end{exercise}}

%\newcounter{ccounter}
%\setcounter{ccounter}{1}
%\newcommand{\Chapter}[1]{\setcounter{chapter}{\arabic{ccounter}}\chapter{#1}\addtocounter{ccounter}{1}}

%\newcommand{\section}[1]{\section{#1}\setcounter{thm}{0}\setcounter{equation}{0}}

%\renewcommand{\theequation}{\arabic{chapter}.\arabic{section}.\arabic{equation}}
%\renewcommand{\thefigure}{\arabic{chapter}.\arabic{figure}}
%\renewcommand{\thetable}{\arabic{chapter}.\arabic{table}}

%\newcommand{\Sec}[2]{\section{#1}\markright{\arabic{ccounter}.\arabic{section}.#2}\setcounter{equation}{0}\setcounter{thm}{0}\setcounter{figure}{0}}
  
\newcommand{\Sec}[2]{\section{#1}}

\setcounter{secnumdepth}{2}
%\setcounter{secnumdepth}{1} 

%\newcounter{THM}
%\renewcommand{\theTHM}{\arabic{chapter}.\arabic{section}}

\newcommand{\trademark}{{R\!\!\!\!\!\bigcirc}}
%\newtheorem{exercise}{}

\newcommand{\dfield}{{\sf SlopeField}}

\newcommand{\pplane}{{\sf PhasePlane}}

\newcommand{\PPLANE}{{\sf PHASEPLANE}}

% BADBAD: \newcommand{\Bbb}{\bf}. % Package amsfonts Warning: Obsolete command \Bbb; \mathbb should be used instead.

\newcommand{\R}{\mbox{$\mathbb{R}$}}
\let\C\relax
\newcommand{\C}{\mbox{$\mathbb{C}$}}
\newcommand{\Z}{\mbox{$\mathbb{Z}$}}
\newcommand{\N}{\mbox{$\mathbb{N}$}}
\newcommand{\D}{\mbox{{\bf D}}}

\newcommand{\WW}{\mathcal{W}}

\usepackage{amssymb}
%\newcommand{\qed}{\hfill\mbox{\raggedright$\square$} \vspace{1ex}}
%\newcommand{\proof}{\noindent {\bf Proof:} \hspace{0.1in}}

\newcommand{\setmin}{\;\mbox{--}\;}
\newcommand{\Matlab}{{M\small{AT\-LAB}} }
\newcommand{\Matlabp}{{M\small{AT\-LAB}}}
\newcommand{\computer}{\Matlab Instructions}
\renewcommand{\computer}{M\small{ATLAB} Instructions}
\newcommand{\half}{\mbox{$\frac{1}{2}$}}
\newcommand{\compose}{\raisebox{.15ex}{\mbox{{\scriptsize$\circ$}}}}
\newcommand{\AND}{\quad\mbox{and}\quad}
\newcommand{\vect}[2]{\left(\begin{array}{c} #1_1 \\ \vdots \\
 #1_{#2}\end{array}\right)}
\newcommand{\mattwo}[4]{\left(\begin{array}{rr} #1 & #2\\ #3
&#4\end{array}\right)}
\newcommand{\mattwoc}[4]{\left(\begin{array}{cc} #1 & #2\\ #3
&#4\end{array}\right)}
\newcommand{\vectwo}[2]{\left(\begin{array}{r} #1 \\ #2\end{array}\right)}
\newcommand{\vectwoc}[2]{\left(\begin{array}{c} #1 \\ #2\end{array}\right)}

\newcommand{\ignore}[1]{}


\newcommand{\inv}{^{-1}}
\newcommand{\CC}{{\cal C}}
\newcommand{\CCone}{\CC^1}
\newcommand{\Span}{{\rm span}}
\newcommand{\rank}{{\rm rank}}
\newcommand{\trace}{{\rm tr}}
\newcommand{\RE}{{\rm Re}}
\newcommand{\IM}{{\rm Im}}
\newcommand{\nulls}{{\rm null\;space}}

\newcommand{\dps}{\displaystyle}
\newcommand{\arraystart}{\renewcommand{\arraystretch}{1.8}}
\newcommand{\arrayfinish}{\renewcommand{\arraystretch}{1.2}}
\newcommand{\Start}[1]{\vspace{0.08in}\noindent {\bf Section~\ref{#1}}}
\newcommand{\exer}[1]{\noindent {\bf \ref{#1}}}
\newcommand{\ans}{\textbf{Answer:} }
\newcommand{\matthree}[9]{\left(\begin{array}{rrr} #1 & #2 & #3 \\ #4 & #5 & #6
\\ #7 & #8 & #9\end{array}\right)}
\newcommand{\cvectwo}[2]{\left(\begin{array}{c} #1 \\ #2\end{array}\right)}
\newcommand{\cmatthree}[9]{\left(\begin{array}{ccc} #1 & #2 & #3 \\ #4 & #5 &
#6 \\ #7 & #8 & #9\end{array}\right)}
\newcommand{\vecthree}[3]{\left(\begin{array}{r} #1 \\ #2 \\
#3\end{array}\right)}
\newcommand{\cvecthree}[3]{\left(\begin{array}{c} #1 \\ #2 \\
#3\end{array}\right)}
\newcommand{\cmattwo}[4]{\left(\begin{array}{cc} #1 & #2\\ #3
&#4\end{array}\right)}

\newcommand{\Matrix}[1]{\ensuremath{\left(\begin{array}{rrrrrrrrrrrrrrrrrr} #1 \end{array}\right)}}

\newcommand{\Matrixc}[1]{\ensuremath{\left(\begin{array}{cccccccccccc} #1 \end{array}\right)}}



\renewcommand{\labelenumi}{\theenumi}
\newenvironment{enumeratea}%
{\begingroup
 \renewcommand{\theenumi}{\alph{enumi}}
 \renewcommand{\labelenumi}{(\theenumi)}
 \begin{enumerate}}
 {\end{enumerate}
 \endgroup}

\newcounter{help}
\renewcommand{\thehelp}{\thesection.\arabic{equation}}

%\newenvironment{equation*}%
%{\renewcommand\endequation{\eqno (\theequation)* $$}%
%   \begin{equation}}%
%   {\end{equation}\renewcommand\endequation{\eqno \@eqnnum
%$$\global\@ignoretrue}}

\author{Martin Golubitsky and Michael Dellnitz}

%\newenvironment{matlabEquation}%
%{\renewcommand\endequation{\eqno (\theequation*) $$}%
%   \begin{equation}}%
%   {\end{equation}\renewcommand\endequation{\eqno \@eqnnum
% $$\global\@ignoretrue}}

\newcommand{\soln}{\textbf{Solution:} }
\newcommand{\exercap}[1]{\centerline{Figure~\ref{#1}}}
\newcommand{\exercaptwo}[1]{\centerline{Figure~\ref{#1}a\hspace{2.1in}
Figure~\ref{#1}b}}
\newcommand{\exercapthree}[1]{\centerline{Figure~\ref{#1}a\hspace{1.2in}
Figure~\ref{#1}b\hspace{1.2in}Figure~\ref{#1}c}}
\newcommand{\para}{\hspace{0.4in}}

%\usepackage{ifluatex}
%\ifluatex
%\ifcsname displaysolutions\endcsname%
%\else
%\renewenvironment{solution}{\suppress}{\endsuppress}
%\fi
%\else
%\renewenvironment{solution}{}{}
%\fi
%
%\ifcsname answer\endcsname
%\renewcommand{\answer}{}
%\fi

%\ifxake
%\newenvironment{matlabEquation}{\begin{equation}}{\end{equation}}
%\else
\newenvironment{matlabEquation}%
{\let\oldtheequation\theequation\renewcommand{\theequation}{\oldtheequation*}\begin{equation}}%
  {\end{equation}\let\theequation\oldtheequation}
%\fi

%\makeatother

\newcommand{\RED}[1]{{\color{red}{#1}}} 


%%
%%
%% Worksheet-builder preamble
%%
%%

\usepackage{xcolor}
\renewenvironment{solution}{\color{blue}}{\color{black}}
\renewenvironment{computerExercise}{\begin{exercise}\textsc{(matlab)} }{\end{exercise}}

%\usepackage{environ}
%\RenewEnviron{prompt}{}
%\RenewEnviron{hint}{}
%\RenewEnviron{multipleChoice}{}
%\RenewEnviron{feedback}{}

\renewcommand{\ans}{\noindent\textbf{Answer: }}
\renewcommand{\soln}{\noindent\textbf{Solution: }}

%\renewcommand{\answer}[2][]{#2}

% if you want to hide solutions, uncomment the following
%\usepackage{comment}\excludecomment{solution}

\def\isitmatlab{}
\newcommand{\matlab}{\def\isitmatlab{ (MATLAB)}}

\makeatletter
\newcommand{\exerciselabel}[2]{\textbf{\textsection #2, Exercise #1\isitmatlab.}\def\@currentlabel{#1}\def\isitmatlab{}}
\makeatother

\newcounter{problemx}
\newcommand{\problemlabel}{\refstepcounter{problemx}\section*{Problem \arabic{problemx}}}
\newcommand{\matlabproblemlabel}{\refstepcounter{problemx}\section*{Problem \arabic{problemx} (MATLAB)}}
%\newcommand{\problemlabel}{\refstepcounter{problem}\section*{Problem}}
%\newcommand{\matlabproblemlabel}{\refstepcounter{problem}\section*{Problem (MATLAB)}}



%\newcommand{\N}{\mathbb N}
%\newcommand{\W}{\mathbb W}
%\newcommand{\C}{\mathbb C}
%\newcommand{\Z}{\mathbb Z}
%\newcommand{\Q}{\mathbb Q}
%\renewcommand{\R}{\mathbb R}

\title{Theorem Review}
\author{Brad Findell}
\begin{document}
\begin{abstract}
Key theorems for Exam 1.
\end{abstract}
\maketitle


\begin{problem}
  Below are some \textbf{augmented matrices} for \textbf{linear
    systems} $Ax = b$ after some row operations have been performed.
  Fill in the blanks.  (Enter NMI for ``need more info.'')
\begin{enumerate}
\item 
  \[
  \left(\begin{array}{ccc|c} 1 & 0 & 0 & 2 \\ 0 & 1 & 0 & 0 \\ 0 & 0 & 0 & 0
  \end{array}\right)
  \]
  Here ${\rm rank}(A) = \answer{2}$, and  ${\rm rank}(A|b) = \answer{2}$.  
  \begin{multipleChoice}
  \choice{There are no solutions.}
  \choice{There is exactly one solution.}
  \choice[correct]{There are infinitely many solutions.}
  \choice{Need more information.}
  \end{multipleChoice}
\item 
  \[
  \left(\begin{array}{ccc|c} 1 & 0 & 0 & 2 \\ 0 & 1 & 0 & 3 \\ 0 & 0 & 0 & 1
  \end{array}\right)
  \]
  Here ${\rm rank}(A) = \answer{2}$, and   ${\rm rank}(A|b) = \answer{3}$.
  \begin{multipleChoice}
  \choice[correct]{There are no solutions.}
  \choice{There is exactly one solution.}
  \choice{There are infinitely many solutions.}
  \choice{Need more information.}
  \end{multipleChoice}
\item 
    \[
  \left(\begin{array}{ccc|c}  1 &  0 &  0 &  6  \\  0 &  1 &  0 &  3  \\  0 &  0 &  1 &  -2  
  \end{array}\right)
  \]
  Here ${\rm rank}(A) = \answer{3}$, and  ${\rm rank}(A|b) = \answer{3}$.
    \begin{multipleChoice}
  \choice{There are no solutions.}
  \choice[choice]{There is exactly one solution.}
  \choice{There are infinitely many solutions.}
  \choice{Need more information.}
  \end{multipleChoice}
\end{enumerate}
 
\end{problem}

\begin{question}
Suppose a linear system of $m$ equations in $n$ unknowns is represented by the matrix equation $Ax=b$ representing $m$ equations in $n$ unknowns, 
and suppose you compute ${\rm rank}(A)$ and ${\rm rank}(A|b)$.  Which of the following situations are possible? 
\begin{selectAll}
  \choice[correct]{${\rm rank}(A) < {\rm rank}(A|b)$.}
  \choice[correct]{${\rm rank}(A) = {\rm rank}(A|b)$.}
  \choice{${\rm rank}(A) > {\rm rank}(A|b)$.}
\end{selectAll}

\begin{question}
Correct!  

When ${\rm rank}(A) < {\rm rank}(A|b)$ what can we conclude? 
\begin{multipleChoice}
  \choice{Not much.  It depends on the circumstances.}
  \choice{The system has a unique solution.}
  \choice{The system has an infinite number of solutions.}
  \choice{The system has at least one solution.}
  \choice[correct]{The system has no solutions.}
\end{multipleChoice}

When ${\rm rank}(A) = {\rm rank}(A|b)$ what can we conclude? 
\begin{multipleChoice}
  \choice{Not much.  It depends on the circumstances.}
  \choice{The system has a unique solution.}
  \choice{The system has an infinite number of solutions.}
  \choice[correct]{The system has at least one solution.}
  \choice{The system has no solutions.}
\end{multipleChoice}

\begin{question}
Correct!  In fact, $Ax=b$ will have the same number of solutions as the homogeneous equation $Ax=0$.  
Furthermore, 
\[
{\rm Number of parameters }= \answer{n} - {\rm rank}(A). 
\]
\end{question}
\end{question}
\end{question}


\begin{problem}
  Let $L_A:{\mathbb R}^3\to {\mathbb R}^2$ be a linear transformation that maps:
  \[
  e_1\mapsto \begin{pmatrix} 1\\ 2 \end{pmatrix},\hspace{1in} 
  e_2\mapsto \begin{pmatrix} -3\\ 1 \end{pmatrix},\hspace{1in} 
  e_3\mapsto \begin{pmatrix} 4\\ 0 \end{pmatrix}.
        \]
        Write the \textbf{matrix} $A$ corresponding to this linear
        transformation.
  \[
  A=\begin{bmatrix} \answer{1} &  \answer{-3} &  \answer{4} \\ 
       \answer{2} &  \answer{1} &  \answer{0}\end{bmatrix}
  \]
        
\end{problem}


\begin{question}
Suppose $A$ is an $m\times n$ matrix.  When thinking of $A$ as a transformation from
\wordChoice{\choice{${\mathbb R}^2$}\choice{${\mathbb R}^3$}\choice{${\mathbb R}^m$}\choice[correct]{${\mathbb R}^n$}} to 
\wordChoice{\choice{${\mathbb R}^2$}\choice{${\mathbb R}^3$}\choice[correct]{${\mathbb R}^m$}\choice{${\mathbb R}^n$}}, the 
$\answer[format=string]{columns}$ of $A$ are the $\answer[format=string]{images}$ of the 
standard basis vectors in ${\mathbb R}^n$. 
\end{question}

\begin{question}
The rank of a matrix is the number of non-zero rows. True or false?  
\begin{multipleChoice}
\choice{True.}
\choice{False.}
\end{multipleChoice}
\begin{question}
Before counting non-zero rows, the matrix must be in $\answer[format=string]{reduced echelon form}$. (Hint: three words without "row.")
\end{question}
\end{question}



%For an 𝑛 × 𝑛 matrix 𝐴, the following are equivalent: 
%1. 𝐴 is invertible. 
%2. The equation 𝐴𝑥 = 𝑏 has a unique solution for each 𝑏 ∈ R𝑛. 
%3. The only solution to 𝐴𝑥=0 is 𝑥=0. 
%4. 𝐴 is row equivalent to 𝐼𝑛. 
%
%Use row operations on (A|I) to find A^{-1}
%
%2 by 2 determinant is area of parallelogram formed by columns
%
%Is a mapping linear? 
%Write matrix for a transformation given images of basis vectors. 
%Is matrix product defined?  Compute when possible. 
%Write inverse using row reduction or formula. 



\end{document}

