\documentclass{ximera}
%\documentclass[space,handout,nooutcomes]{ximera}

%\input{../preamble.tex}

%\newcommand{\N}{\mathbb N}
%\newcommand{\W}{\mathbb W}
%\newcommand{\C}{\mathbb C}
%\newcommand{\Z}{\mathbb Z}
%\newcommand{\Q}{\mathbb Q}
%\renewcommand{\R}{\mathbb R}

\title{Theorem Review}
\author{Brad Findell}
\begin{document}
\begin{abstract}
Key theorems for Exam 1.
\end{abstract}
\maketitle


\begin{problem}
  Below are some \textbf{augmented matrices} for \textbf{linear
    systems} $Ax = b$ after some row operations have been performed.
  Fill in the blanks.  (Enter NMI for ``need more info.'')
\begin{enumerate}
\item 
  \[
  \left(\begin{array}{ccc|c} 1 & 0 & 0 & 2 \\ 0 & 1 & 0 & 0 \\ 0 & 0 & 0 & 0
  \end{array}\right)
  \]
  Here ${\rm rank}(A) = \answer{2}$, and  ${\rm rank}(A|b) = \answer{2}$.  
  \begin{multipleChoice}
  \choice{There are no solutions.}
  \choice{There is exactly one solution.}
  \choice{There are infinitely many solutions.}
  \choice{Need more information.}
  \end{multipleChoice}
\item 
  \[
  \left(\begin{array}{ccc|c} 1 & 0 & 0 & 2 \\ 0 & 1 & 0 & 3 \\ 0 & 0 & 0 & 1
  \end{array}\right)
  \]
  Here ${\rm rank}(A) = \answer{2}$, and   ${\rm rank}(A|b) = \answer{3}$.
  \begin{multipleChoice}
  \choice{There are no solutions.}
  \choice{There is exactly one solution.}
  \choice{There are infinitely many solutions.}
  \choice{Need more information.}
  \end{multipleChoice}
\item 
    \[
  \left(\begin{array}{ccc|c}  1 &  0 &  0 &  6  \\  0 &  1 &  0 &  3  \\  0 &  0 &  1 &  -2  
  \end{array}\right)
  \]
  Here ${\rm rank}(A) = \answer{3}$, and  ${\rm rank}(A|b) = \answer{3}$.
    \begin{multipleChoice}
  \choice{There are no solutions.}
  \choice{There is exactly one solution.}
  \choice{There are infinitely many solutions.}
  \choice{Need more information.}
  \end{multipleChoice}
\end{enumerate}
 
\end{problem}

\begin{question}
Suppose a linear system of $m$ equations in $n$ unknowns is represented by the matrix equation $Ax=b$ representing $m$ equations in $n$ unknowns, 
and suppose you compute ${\rm rank}(A)$ and ${\rm rank}(A|b)$.  Which of the following situations are possible? 
\begin{selectAll}
  \choice[correct]{${\rm rank}(A) < {\rm rank}(A|b)$.}
  \choice[correct]{${\rm rank}(A) = {\rm rank}(A|b)$.}
  \choice{${\rm rank}(A) > {\rm rank}(A|b)$.}
\end{selectAll}

\begin{question}
Correct!  

When ${\rm rank}(A) < {\rm rank}(A|b)$ what can we conclude? 
\begin{multipleChoice}
  \choice{Not much.  It depends on the circumstances.}
  \choice{The system has a unique solution.}
  \choice{The system has an infinite number of solutions.}
  \choice{The system has at least one solution.}
  \choice[correct]{The system has no solutions.}
\end{multipleChoice}

When ${\rm rank}(A) = {\rm rank}(A|b)$ what can we conclude? 
\begin{multipleChoice}
  \choice{Not much.  It depends on the circumstances.}
  \choice{The system has a unique solution.}
  \choice{The system has an infinite number of solutions.}
  \choice[correct]{The system has at least one solution.}
  \choice{The system has no solutions.}

\begin{question}
Correct!  In fact, $Ax=b$ will have the same number of solutions as the homogeneous equation $Ax=0$.  
Furthermore, 
\[
{\rm Number of parameters }= \answer{n} - {\rm rank}(A). 
\]
\end{question}
\end{multipleChoice}
\end{question}
\end{question}


\begin{problem}
  Let $L_A:{\mathbb R}^3\to {\mathbb R}^2$ be a linear transformation that maps:
  \[
  e_1\mapsto \begin{pmatrix} 1\\ 2 \end{pmatrix},\hspace{1in} 
  e_2\mapsto \begin{pmatrix} -3\\ 1 \end{pmatrix},\hspace{1in} 
  e_3\mapsto \begin{pmatrix} 4\\ 0 \end{pmatrix}.
        \]
        Write the \textbf{matrix} $A$ corresponding to this linear
        transformation.
  \[
  A=\begin{bmatrix} \answer{1} &  \answer{-3} &  \answer{4} \\ 
       \answer{2} &  \answer{1} &  \answer{0}\end{bmatrix}
  \]
        
\end{problem}


\begin{question}
Suppose $A$ is an $m\times n$ matrix.  When thinking of $A$ as a transformation from
\wordChoice{\choice{${\mathbb R}^2$}\choice{${\mathbb R}^3$}\choice{${\mathbb R}^m$}\choice[correct]{${\mathbb R}^n$}} to 
\wordChoice{\choice{${\mathbb R}^2$}\choice{${\mathbb R}^3$}\choice[correct]{${\mathbb R}^m$}\choice{${\mathbb R}^n$}}, the 
$\answer[format=string]{columns}$ of $A$ are the $\answer[format=string]{images}$ of the 
standard basis vectors in ${\mathbb R}^n$. 
\end{question}

\begin{question}
The rank of a matrix is the number of non-zero rows. True or false?  
\begin{multipleChoice}
\choice{True.}
\choice{False.}
\end{multipleChoice}
\begin{question}
Before counting non-zero rows, the matrix must be in $\answer[format=string]{reduced echelon form}$. (Hint: three words without "row.")
\end{question}
\end{question}



%For an 𝑛 × 𝑛 matrix 𝐴, the following are equivalent: 
%1. 𝐴 is invertible. 
%2. The equation 𝐴𝑥 = 𝑏 has a unique solution for each 𝑏 ∈ R𝑛. 
%3. The only solution to 𝐴𝑥=0 is 𝑥=0. 
%4. 𝐴 is row equivalent to 𝐼𝑛. 
%
%Use row operations on (A|I) to find A^{-1}
%
%2 by 2 determinant is area of parallelogram formed by columns
%
%Is a mapping linear? 
%Write matrix for a transformation given images of basis vectors. 
%Is matrix product defined?  Compute when possible. 
%Write inverse using row reduction or formula. 



\end{document}

