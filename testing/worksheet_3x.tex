

\documentclass{article}

\usepackage{pgf,tikz}
\usepackage{mathrsfs}
\usetikzlibrary{shapes,arrows}
\usepackage{framed}
\pgfplotsset{compat=1.13}

\graphicspath{
  {./}
  {ximeraTutorial/}
}

\newenvironment{sectionOutcomes}{}{}


% Below is the preamble from the textbook in /laode, by Golubitsky and Dellnitz
% ... with problematic stuff commented out
% Followed by the hw-preamble from worksheet-builder

%\usepackage{ulem}
\usepackage[normalem]{ulem}

\epstopdfsetup{outdir=./}

\usepackage{morewrites}
\makeatletter
\newcommand\subfile[1]{%
\renewcommand{\input}[1]{}%
\begingroup\skip@preamble\otherinput{#1}\endgroup\par\vspace{\topsep}
\let\input\otherinput}
\makeatother

\newcommand{\EXER}{}
\newcommand{\includeexercises}{\EXER\directlua{dofile(kpse.find_file("exercises","lua"))}}

\newenvironment{computerExercise}{\begin{exercise}}{\end{exercise}}

%\newcounter{ccounter}
%\setcounter{ccounter}{1}
%\newcommand{\Chapter}[1]{\setcounter{chapter}{\arabic{ccounter}}\chapter{#1}\addtocounter{ccounter}{1}}

%\newcommand{\section}[1]{\section{#1}\setcounter{thm}{0}\setcounter{equation}{0}}

%\renewcommand{\theequation}{\arabic{chapter}.\arabic{section}.\arabic{equation}}
%\renewcommand{\thefigure}{\arabic{chapter}.\arabic{figure}}
%\renewcommand{\thetable}{\arabic{chapter}.\arabic{table}}

%\newcommand{\Sec}[2]{\section{#1}\markright{\arabic{ccounter}.\arabic{section}.#2}\setcounter{equation}{0}\setcounter{thm}{0}\setcounter{figure}{0}}
  
\newcommand{\Sec}[2]{\section{#1}}

\setcounter{secnumdepth}{2}
%\setcounter{secnumdepth}{1} 

%\newcounter{THM}
%\renewcommand{\theTHM}{\arabic{chapter}.\arabic{section}}

\newcommand{\trademark}{{R\!\!\!\!\!\bigcirc}}
%\newtheorem{exercise}{}

\newcommand{\dfield}{{\sf SlopeField}}

\newcommand{\pplane}{{\sf PhasePlane}}

\newcommand{\PPLANE}{{\sf PHASEPLANE}}

% BADBAD: \newcommand{\Bbb}{\bf}. % Package amsfonts Warning: Obsolete command \Bbb; \mathbb should be used instead.

\newcommand{\R}{\mbox{$\mathbb{R}$}}
\let\C\relax
\newcommand{\C}{\mbox{$\mathbb{C}$}}
\newcommand{\Z}{\mbox{$\mathbb{Z}$}}
\newcommand{\N}{\mbox{$\mathbb{N}$}}
\newcommand{\D}{\mbox{{\bf D}}}

\newcommand{\WW}{\mathcal{W}}

\usepackage{amssymb}
%\newcommand{\qed}{\hfill\mbox{\raggedright$\square$} \vspace{1ex}}
%\newcommand{\proof}{\noindent {\bf Proof:} \hspace{0.1in}}

\newcommand{\setmin}{\;\mbox{--}\;}
\newcommand{\Matlab}{{M\small{AT\-LAB}} }
\newcommand{\Matlabp}{{M\small{AT\-LAB}}}
\newcommand{\computer}{\Matlab Instructions}
\renewcommand{\computer}{M\small{ATLAB} Instructions}
\newcommand{\half}{\mbox{$\frac{1}{2}$}}
\newcommand{\compose}{\raisebox{.15ex}{\mbox{{\scriptsize$\circ$}}}}
\newcommand{\AND}{\quad\mbox{and}\quad}
\newcommand{\vect}[2]{\left(\begin{array}{c} #1_1 \\ \vdots \\
 #1_{#2}\end{array}\right)}
\newcommand{\mattwo}[4]{\left(\begin{array}{rr} #1 & #2\\ #3
&#4\end{array}\right)}
\newcommand{\mattwoc}[4]{\left(\begin{array}{cc} #1 & #2\\ #3
&#4\end{array}\right)}
\newcommand{\vectwo}[2]{\left(\begin{array}{r} #1 \\ #2\end{array}\right)}
\newcommand{\vectwoc}[2]{\left(\begin{array}{c} #1 \\ #2\end{array}\right)}

\newcommand{\ignore}[1]{}


\newcommand{\inv}{^{-1}}
\newcommand{\CC}{{\cal C}}
\newcommand{\CCone}{\CC^1}
\newcommand{\Span}{{\rm span}}
\newcommand{\rank}{{\rm rank}}
\newcommand{\trace}{{\rm tr}}
\newcommand{\RE}{{\rm Re}}
\newcommand{\IM}{{\rm Im}}
\newcommand{\nulls}{{\rm null\;space}}

\newcommand{\dps}{\displaystyle}
\newcommand{\arraystart}{\renewcommand{\arraystretch}{1.8}}
\newcommand{\arrayfinish}{\renewcommand{\arraystretch}{1.2}}
\newcommand{\Start}[1]{\vspace{0.08in}\noindent {\bf Section~\ref{#1}}}
\newcommand{\exer}[1]{\noindent {\bf \ref{#1}}}
\newcommand{\ans}{\textbf{Answer:} }
\newcommand{\matthree}[9]{\left(\begin{array}{rrr} #1 & #2 & #3 \\ #4 & #5 & #6
\\ #7 & #8 & #9\end{array}\right)}
\newcommand{\cvectwo}[2]{\left(\begin{array}{c} #1 \\ #2\end{array}\right)}
\newcommand{\cmatthree}[9]{\left(\begin{array}{ccc} #1 & #2 & #3 \\ #4 & #5 &
#6 \\ #7 & #8 & #9\end{array}\right)}
\newcommand{\vecthree}[3]{\left(\begin{array}{r} #1 \\ #2 \\
#3\end{array}\right)}
\newcommand{\cvecthree}[3]{\left(\begin{array}{c} #1 \\ #2 \\
#3\end{array}\right)}
\newcommand{\cmattwo}[4]{\left(\begin{array}{cc} #1 & #2\\ #3
&#4\end{array}\right)}

\newcommand{\Matrix}[1]{\ensuremath{\left(\begin{array}{rrrrrrrrrrrrrrrrrr} #1 \end{array}\right)}}

\newcommand{\Matrixc}[1]{\ensuremath{\left(\begin{array}{cccccccccccc} #1 \end{array}\right)}}



\renewcommand{\labelenumi}{\theenumi}
\newenvironment{enumeratea}%
{\begingroup
 \renewcommand{\theenumi}{\alph{enumi}}
 \renewcommand{\labelenumi}{(\theenumi)}
 \begin{enumerate}}
 {\end{enumerate}
 \endgroup}

\newcounter{help}
\renewcommand{\thehelp}{\thesection.\arabic{equation}}

%\newenvironment{equation*}%
%{\renewcommand\endequation{\eqno (\theequation)* $$}%
%   \begin{equation}}%
%   {\end{equation}\renewcommand\endequation{\eqno \@eqnnum
%$$\global\@ignoretrue}}

\author{Martin Golubitsky and Michael Dellnitz}

%\newenvironment{matlabEquation}%
%{\renewcommand\endequation{\eqno (\theequation*) $$}%
%   \begin{equation}}%
%   {\end{equation}\renewcommand\endequation{\eqno \@eqnnum
% $$\global\@ignoretrue}}

\newcommand{\soln}{\textbf{Solution:} }
\newcommand{\exercap}[1]{\centerline{Figure~\ref{#1}}}
\newcommand{\exercaptwo}[1]{\centerline{Figure~\ref{#1}a\hspace{2.1in}
Figure~\ref{#1}b}}
\newcommand{\exercapthree}[1]{\centerline{Figure~\ref{#1}a\hspace{1.2in}
Figure~\ref{#1}b\hspace{1.2in}Figure~\ref{#1}c}}
\newcommand{\para}{\hspace{0.4in}}

%\usepackage{ifluatex}
%\ifluatex
%\ifcsname displaysolutions\endcsname%
%\else
%\renewenvironment{solution}{\suppress}{\endsuppress}
%\fi
%\else
%\renewenvironment{solution}{}{}
%\fi
%
%\ifcsname answer\endcsname
%\renewcommand{\answer}{}
%\fi

%\ifxake
%\newenvironment{matlabEquation}{\begin{equation}}{\end{equation}}
%\else
\newenvironment{matlabEquation}%
{\let\oldtheequation\theequation\renewcommand{\theequation}{\oldtheequation*}\begin{equation}}%
  {\end{equation}\let\theequation\oldtheequation}
%\fi

%\makeatother

\newcommand{\RED}[1]{{\color{red}{#1}}} 


%%
%%
%% Worksheet-builder preamble
%%
%%

\usepackage{xcolor}
\renewenvironment{solution}{\color{blue}}{\color{black}}
\renewenvironment{computerExercise}{\begin{exercise}\textsc{(matlab)} }{\end{exercise}}

%\usepackage{environ}
%\RenewEnviron{prompt}{}
%\RenewEnviron{hint}{}
%\RenewEnviron{multipleChoice}{}
%\RenewEnviron{feedback}{}

\renewcommand{\ans}{\noindent\textbf{Answer: }}
\renewcommand{\soln}{\noindent\textbf{Solution: }}

%\renewcommand{\answer}[2][]{#2}

% if you want to hide solutions, uncomment the following
%\usepackage{comment}\excludecomment{solution}

\def\isitmatlab{}
\newcommand{\matlab}{\def\isitmatlab{ (MATLAB)}}

\makeatletter
\newcommand{\exerciselabel}[2]{\textbf{\textsection #2, Exercise #1\isitmatlab.}\def\@currentlabel{#1}\def\isitmatlab{}}
\makeatother

\newcounter{problemx}
\newcommand{\problemlabel}{\refstepcounter{problemx}\section*{Problem \arabic{problemx}}}
\newcommand{\matlabproblemlabel}{\refstepcounter{problemx}\section*{Problem \arabic{problemx} (MATLAB)}}
%\newcommand{\problemlabel}{\refstepcounter{problem}\section*{Problem}}
%\newcommand{\matlabproblemlabel}{\refstepcounter{problem}\section*{Problem (MATLAB)}}



% Comment out next line to exclude solutions 

%\usepackage{comment}\excludecomment{solution}\title{Homework 3}\author{Math 2568}\date{Due: September 10, 2018}
\title{Homework 3}
\author{Dr. Golubitsky}
\date{Due: Wednesday, September 16, 2020 \@9:10am}


%\makeatletter
%\newlabel{c2.4.1}{{1}{92}}
%\newlabel{c2.4.1b}{{2}{93}}
%\newlabel{eq:avect}{{3.1.6}{122}}
%\newlabel{c4.6.-1a}{{1}{169}}
%\newlabel{c4.6.-1d}{{4}{169}}
%\newlabel{lin-matrices}{{3.3.5}{142}}
%\newlabel{columnsA}{{3.3.4}{142}}
%\newlabel{c4.1.a10a}{{14}{123}}
%\newlabel{c4.1.a10b}{{15}{123}}
%\newlabel{c4.3.6a}{{6}{145}}
%\newlabel{c4.3.6d}{{9}{146}}
%\newlabel{sum}{{3.3.1}{138}}
%\newlabel{product}{{3.3.2}{138}}
%\newlabel{c4.2.a1a}{{1}{130}}
%\newlabel{c4.2.a1c}{{3}{131}}
%\makeatother
\begin{document}
\maketitle


\problemlabel

\noindent In Exercises~\ref{c2.4.1} -- \ref{c2.4.1b} row reduce the given 
matrix to reduced echelon form by hand and determine its rank.


\exerciselabel{2}{2.4}\begin{exercise} \label{c2.4.1b}
$B=\left(\begin{array}{rrr}
1 &  -2 & 3\\
3 &  -6 & 9 \\
1 &  -8 & 2
         \end{array}\right)$
  
\begin{solution}
\soln
The reduced echelon form of the matrix is:
\[
B=\left(\begin{array}{rrc} 1 & 0 & 10/3 \\ 
0 & 1 & 1/6 \\ 0 & 0 & 0 \end{array}\right)
\]
The rank of $B$ is $2$, since the reduced echelon matrix has two nonzero
rows.

\end{solution}
\end{exercise}


%%%%%%%%%%%%%%%%%%%%%%%%%%%%%%%%%%%%%%%%%%%%%%%%%%%%%%%%%%%%%%%%



\problemlabel

% To which section does your exercise belong? 


\exerciselabel{6}{2.4}\begin{exercise}\label{c2.4.2b.2}

Consider the system of equations
\[
\begin{array}{rcl}
x_1 + 3x_3 & = & 1 \\
-x_1+2x_2-3x_3 & = & 1\\
2x_2 + ax_3 & = & b
\end{array}
\]
Find all pairs of real numbers $a$ and $b$ where the system has no solutions, a unique solution, or infinitely many solutions?  Your answer should subdivide the $ab$-plane into three disjoint sets.
  
\begin{solution}

\ans Unique solutions occur when $a\neq 0$; no solution occurs when $a=0$ and $b\neq 2$; and infinitely many solutions exist when $a = 0$ and $b =2$.

\soln 
Use row reduction on the augmented matrix to obtain
\[
\Matrix{1 & 0 & 3 & 1\\ -1 & 2 & -3 & 1 \\ 0 & 2 & a & b} \to
\Matrix{1 & 0 & 3 & 1\\  0 & 2 & 0 & 2 \\ 0 & 2 & a & b} \to
\]
\[
\Matrixc{1 & 0 & 3 & 1\\  0 & 2 & 0 & 2 \\ 0 & 0 & a & b - 2} \to
\Matrixc{1 & 0 & 3 & 1\\  0 & 1 & 0 & 1 \\ 0 & 0 & a & b - 2} 
\]
If $a\neq 0$ the system has a unique solution. If $a = 0$ we obtain the echelon form matrix
\[
\Matrixc{1 & 0 & 3 & 1\\  0 & 1 & 0 & 1 \\ 0 & 0 & 0 & b - 2}
\]  
There are no solutions if $b\neq 2$ and infinitely many solutions if $b = 2$.

\end{solution}
\end{exercise}


%%%%%%%%%%%%%%%%%%%%%%%%%%%%%%%%%%%%%%%%%%%%%%%%%%%%%%%%%%%%%%%%



\problemlabel



\exerciselabel{11}{3.1}\begin{exercise} \label{c4.1.7}
Let $A$ be a $2\times 2$ matrix.  Find $A$ so that
\begin{eqnarray*}
A\left(\begin{array}{c} 1 \\ 0 \end{array}\right) =
\left(\begin{array}{r} 3 \\ -5 \end{array}\right) \\
A\left(\begin{array}{c} 0 \\ 1 \end{array}\right) =
\left(\begin{array}{r} 1 \\ 4 \end{array}\right).
\end{eqnarray*}

\begin{solution}

\ans The equations are valid when
\[ A = \left(\begin{array}{rr} 3 & 1 \\ -5 & 4\end{array}\right). \]
\soln Let
\[ A = \left(\begin{array}{rr} a_{11} & a_{12} \\ a_{21} & 
a_{22}\end{array}\right). \]
So
\[ \left(\begin{array}{rr} a_{11} & a_{12} \\ a_{21} & a_{22}\end{array}\right)
\left(\begin{array}{r} 1 \\ 0\end{array}\right) =
\left(\begin{array}{r} 3 \\ -5\end{array}\right) \AND
\left(\begin{array}{rr} a_{11} & a_{12} \\ a_{21} & a_{22}\end{array}\right)
\left(\begin{array}{r} 0 \\ 1\end{array}\right) =
\left(\begin{array}{r} 1 \\ 4.\end{array}\right) \]
These matrix equations are equivalent to the linear equations
\[ \begin{array}{rcl}
a_{11} & = & 3 \\
a_{21} & = & -5 \\
a_{12} & = & 1 \\
a_{22} & = & 4.\end{array} \]


\end{solution}
\end{exercise}


%%%%%%%%%%%%%%%%%%%%%%%%%%%%%%%%%%%%%%%%%%%%%%%%%%%%%%%%%%%%%%%%



\problemlabel

\exerciselabel{13}{3.1}\begin{exercise} \label{c4.1.9}
Is there an upper triangular $2\times 2$ matrix $A$ such that
\begin{equation}  \label{eq:avect}
A\vectwo{1}{0} = \vectwo{1}{2}?
\end{equation}
Is there a symmetric $2\times 2$ matrix $A$ satisfying \eqref{eq:avect}?

\begin{solution}

\ans There is no $2 \times 2$ upper triangular matrix $A$ that
satisfies equation \eqref{eq:avect}, but any symmetric matrix $A$ of the form
\[ A = \mattwo{1}{2}{2}{a_{22}}, \]
where $a_{22}$ is a real number, satisfies \eqref{eq:avect}.

\soln Let $A$ be the upper triangular matrix
\[ \mattwo{a_{11}}{a_{12}}{0}{a_{22}}. \]
The resulting matrix equation
\[ \mattwo{a_{11}}{a_{12}}{0}{a_{22}}
\vectwo{1}{0} = \vectwo{1}{2} \]
yields the linear equations
\[ \begin{array}{rcl}
a_{11} & = & 1 \\
0 & = & 2.\end{array} \]
The second equation is inconsistent, so there is no solution.

\para Then let $A$ be the symmetric matrix
\[ \mattwo{a_{11}}{a_{12}}{a_{12}}{a_{22}}. \]
Write the matrix equation
\[ \mattwo{a_{11}}{a_{12}}{a_{12}}{a_{22}}
\vectwo{1}{0} = \vectwo{1}{2}, \]
from which we obtain the consistent linear system
\[ \begin{array}{rcl}
a_{11} & = & 1 \\
a_{12} & = & 2.\end{array} \]

\end{solution}
\end{exercise}


%%%%%%%%%%%%%%%%%%%%%%%%%%%%%%%%%%%%%%%%%%%%%%%%%%%%%%%%%%%%%%%%



\problemlabel

\exerciselabel{6}{3.2}\begin{exercise} \label{c4.2.1c}
What $2\times 2$ matrix rotates the plane clockwise by $90^\circ$ while
dilating it by a factor of $2$?

\begin{solution}
\ans
\[
2R_{(-90^\circ)} = \mattwo{2\cos (-90^\circ)}{-2\sin
(-90^\circ)}{2\sin (-90^\circ)}{2\cos (-90^\circ)} = \mattwo{0}{2}{-2}{0}.
\]

\end{solution}
\end{exercise}


%%%%%%%%%%%%%%%%%%%%%%%%%%%%%%%%%%%%%%%%%%%%%%%%%%%%%%%%%%%%%%%%



\problemlabel

\exerciselabel{13}{3.2}\begin{exercise} \label{c4.2.3}
Find a $2\times 3$ matrix $P$ that projects three dimensional $xyz$ space onto
the $xy$ plane.  {\bf Hint:} Such a matrix will satisfy
\[
P\left(\begin{array}{c} 0 \\ 0 \\ z \end{array}\right) = \vectwo{0}{0}
\AND
P\left(\begin{array}{c} x \\ y \\ 0 \end{array}\right) = \vectwo{x}{y}.
\]

\begin{solution}

\ans The matrix is
\[ P = \left(\begin{array}{rrr} 1 & 0 & 0 \\ 0 & 1 & 0\end{array}\right). \]

\soln Let
\[ P = \left(\begin{array}{rrr} p_{11} & p_{12} & p_{13} \\
p_{21} & p_{22} & p_{23}\end{array}\right). \]
Note that a matrix that projects $xyz$ space onto the $xy$ plane
satisfies the vector equations:
\[ \begin{array}{c} \left(\begin{array}{rrr} p_{11} & p_{12} &
p_{13} \\ p_{21} & p_{22} & p_{23} \end{array}\right)
\vecthree{1}{0}{0} = \vectwo{1}{0} \\
\left(\begin{array}{rrr} p_{11} & p_{12} & p_{13} \\ p_{21} &
p_{22} & p_{23} \end{array}\right) \vecthree{0}{1}{0} = \vectwo{0}{1} \\
\left(\begin{array}{rrr} p_{11} & p_{12} & p_{13} \\ p_{21} & p_{22}
& p_{23} \end{array}\right) \vecthree{0}{0}{1} = \vectwo{0}{0} \end{array} 
\]
from which we get the equations
\[ \begin{array}{rrrrrrr}
p_{11} & = & 1 \\
p_{21} & = & 0 \end{array},
\quad
\begin{array}{rrrrrrr}
p_{12} & = & 0 \\
p_{22} & = & 1 \end{array}
\AND
\begin{array}{rrrrrrr}
p_{13} & = & 0 \\
p_{23} & = & 0 \end{array}. 
\]
Substitute these values back into $P$ to obtain the solution.

\end{solution}
\end{exercise}


%%%%%%%%%%%%%%%%%%%%%%%%%%%%%%%%%%%%%%%%%%%%%%%%%%%%%%%%%%%%%%%%



\problemlabel

\exerciselabel{5}{3.3}\begin{exercise} \label{c4.3.5}
Let $x=(3,-2)$, $y=(2,3)$, and $z=(1,4)$.  For which real
numbers $\alpha,\beta,\gamma$ does
\[
\alpha x + \beta y + \gamma z = (1,-2)?
\]

\begin{solution}

\ans The equation
\[ \alpha(3,-2) + \beta(2,3) + \gamma(1,4) = (1,-2) \]
holds for any real numbers $\alpha$,$\beta$,
$\gamma$ such that $\alpha = \frac{5}{13}\gamma +
\frac{7}{13}$ and $\beta = -\frac{14}{13}\gamma
- \frac{4}{13}$.

\soln Write the equation as the linear system
\[ \begin{array}{rrrrrrl}
3\alpha & + & 2\beta & + & \gamma & = & 1 \\
-2\alpha & + & 3\beta & + & 4\gamma & = & -2. \end{array} \]
The augmented matrix
\[ \left(\begin{array}{rrr|r}
3 & 2 & 1 & 1 \\
-2 & 3 & 4 & -2 \end{array}\right) \]
row reduces to
\[ \left(\begin{array}{rrr|r}
1 & 0 & -\frac{5}{13} & \frac{7}{13} \\
0 & 1 & \frac{14}{13} & -\frac{4}{13} \end{array}\right) \]
and the equation is valid for any values of $\alpha$,$\beta$,
and $\gamma$ that satisfy this system.

\end{solution}
\end{exercise}


%%%%%%%%%%%%%%%%%%%%%%%%%%%%%%%%%%%%%%%%%%%%%%%%%%%%%%%%%%%%%%%%



\problemlabel

\exerciselabel{10}{3.3}\begin{exercise}\label{c4.3.6A}

Determine which of the following maps are linear maps.  If the map is linear give the matrix associated to the linear map. Explain your reasoning.
\begin{enumeratea}

\item $L_1:\R^2\to\R^2$ where $L_1\Matrix{x\\y} = \Matrixc{x + y+ 3\\ 2y +1}$

\item $L_2:\R^2\to\R^3$ where $L_2\Matrix{x\\y} = \Matrixc{\sin x\\ x + y\\ 2y}$
 
\item $L_3:\R^2\to\R$ where $L_3\Matrix{x\\y} = x + y $
  
\end{enumeratea}
  
\begin{solution}

\ans (a) Not linear; (b) not linear; (c) linear with $1\times 2$ matrix $A= \Matrix{1 & 1}$.

\soln 
\begin{enumeratea}

\item Linear maps map the origin to the origin.  $L_1\Matrix{0\\0} = \Matrix{3\\1} \neq 0$. So $L_1$ is not linear.

\item Linear maps $L$ satisfy $L(cX) = cL(X)$.  In this case 
\[
cL_2\Matrix{x\\y} =  \Matrixc{c\sin x\\ cx + cy\\ 2cy} \AND
L_2\left(\Matrix{cx\\cy}\right) =  \Matrixc{\sin(cx)\\ cx + cy\\ 2cy}
\]
Since $\sin(cx)\neq c\sin(x)$, $L_3$ is not linear.

\item All matrix mappings are linear.  Since we can write
\[
L_3 \Matrix{x\\y} = x+ y = \Matrix{1 &1} \Matrix{x\\y}  
\]
it follows that $L_3$ is linear with $1\times 2$ matrix $A =  \Matrix{1 &1}$.
\end{enumeratea}


\end{solution}
\end{exercise}


%%%%%%%%%%%%%%%%%%%%%%%%%%%%%%%%%%%%%%%%%%%%%%%%%%%%%%%%%%%%%%%%




\problemlabel

\noindent In Exercises~\ref{c4.6.-1a} -- \ref{c4.6.-1d} determine whether or 
not the matrix products $AB$ or $BA$ can be computed for each given pair of 
matrices $A$ and $B$.  If the product is possible, perform the computation.


\exerciselabel{2}{3.5}\begin{exercise}  \label{c4.6.-1b}
$A=\left(\begin{array}{rrr} 0 & -2 & 1\\ 4 & 10 & 0 \end{array}\right)$
and $B=\left(\begin{array}{rr} 0 & 2 \\ 3 & -1 \end{array}\right)$.

\begin{solution}
\ans $AB$ is not defined. $BA=\left(\begin{array}{rrr} 8 &  20 &  0\\
 -4 & -16  &  3\end{array}\right)$
\end{solution}
\end{exercise}


%%%%%%%%%%%%%%%%%%%%%%%%%%%%%%%%%%%%%%%%%%%%%%%%%%%%%%%%%%%%%%%%



\problemlabel

\exerciselabel{10}{3.5}\begin{exercise} \label{c4.6.2}
Let
\[
A=\mattwo{2}{5}{1}{4} \AND B=\mattwo{a}{3}{b}{2}.
\]
For which values of $a$ and $b$ does $AB=BA$?

\begin{solution}

\ans 
\[
B = \mattwo{\frac{4}{5}}{3}{\frac{3}{5}}{2}.
\]

\soln Compute
\[
\begin{array}{rcl}
AB & = & BA \\
\mattwo{2}{5}{1}{4}\mattwo{a}{3}{b}{2} & = &
\mattwo{a}{3}{b}{2}\mattwo{2}{5}{1}{4} \\
\mattwo{2a+5b}{16}{a+4b}{11} & = & \mattwo{2a+3}{5a+12}{2b+2}{5b+8}.
\end{array}
\]
This equation can be rewritten as the system
\[
\begin{array}{rcl}
2a + 5b & = & 2a + 3 \\
16 & = & 5a + 12 \\
a + 4b & = & 2b + 2 \\
11 & = & 5b + 8 \end{array}
\]
which yields the solution $a = 4/5$ and $b = 3/5$.

\end{solution}
\end{exercise}


%%%%%%%%%%%%%%%%%%%%%%%%%%%%%%%%%%%%%%%%%%%%%%%%%%%%%%%%%%%%%%%%


\end{document}




\problemlabel

\exerciselabel{16}{3.3}\begin{exercise}  \label{c4.3.12}
Let $P:\R^n\to\R^m$ and $Q:\R^n\to\R^m$ be linear mappings. 
\begin{enumeratea}
\item Prove that $S:\R^n\to\R^m$ defined by
\[
S(x) = P(x) + Q(x)
\]
is also a linear mapping.  
\item Theorem~\ref{lin-matrices} states that there are matrices $A$, $B$ and $C$ such that
\[
P = L_A \AND Q = L_B \AND S = L_C .
\]
What is the relationship between the matrices $A$, $B$, and $C$?
\end{enumeratea}

\begin{solution}

\soln The mapping $L$ is linear if $L(x + y) = L(x) + L(y)$ and if $cL(x) = L(cx)$.  
\begin{enumeratea}
\item We can use the assumption that $P(x)$ and $Q(x)$ are linear mappings to show:
\[ 
\begin{array}{rcl}
S(x + y) & = & P(x + y) + Q(x + y) \\
& = & P(x) + P(y) + Q(x) + Q(y) \\
& = & [P(x) + Q(x)] + [P(y) + Q(y)] \\
& = & S(x) + S(y) 
\end{array} 
\]
and
\[ 
\begin{array}{rcl}
cS(x) & = & cP(x) + cQ(x) \\
& = & P(cx) + Q(cx) \\
& = & S(cx). 
\end{array} 
\]

\item Assume that $S = L_C$, $P = L_A$ and $Q = L_B$ for
$m \times n$ matrices $A$, $B$, $C$.  We claim that
$A = B + C$.  By definition, $A(e_j) = L_A(e_j) =  L_B(e_j) + L_C(e_j) = (B+C)(e_j)$.  
Lemma~\ref{columnsA} implies that the $j^{th}$ column of 
$C$ is the sum of the $j^{th}$ column of $A$  and the $j^{th}$ column of $B$ 
for all columns $j$, so $C = A + B$.
\end{enumeratea}
\end{solution}
\end{exercise}


%%%%%%%%%%%%%%%%%%%%%%%%%%%%%%%%%%%%%%%%%%%%%%%%%%%%%%%%%%%%%%%%



% 2.4 #2, #4
% 2.5 #1
% 3.1 # 5, 7, 11
	% 3.1 #6
% 3.2 # 3, 6
	% 3.2 # 12, # 8
% 3.3 # 5, 7, 8, 11


\matlabproblemlabel

\noindent In Exercises~\ref{c4.1.a10a} -- \ref{c4.1.a10b} use \Matlab to
compute $b=Ax$ for the given $A$ and $x$.


\exerciselabel{15}{3.1}\begin{computerExercise} \label{c4.1.a10b}
\begin{matlabEquation}\label{multiplication-exercise-2}
A=\left(
\begin{array}{rrrrrrr}
    14 & -22  &-26 &  -2 & -77 & 100 & -90\\
    26 &  25  &-15 & -63 &  33 &  92 &  14\\
   -53 &  40  & 19 &  40 & -27 & -88 &  40\\
    10 & -21  & 13 &  97 & -72 & -28 &  92\\
    86 & -17  & 43 &  61 &  13 &  10 &  50\\
   -33 &  31  &  2 &  41 &  65 & -48 &  48\\
    31 &  68  & 55 &  -3 &  35 &  19 & -14
\end{array}
\right)\end{matlabEquation}
and
\begin{equation*}
x = \left( \begin{array}{r} 2.7\\   6.1\\   -8.3\\    8.9\\    8.3\\    2\\
  -4.9
\end{array}\right).
\end{equation*}

\begin{solution}
Load the system into \Matlabp, then type {\tt b = A*x}
to obtain:
\begin{verbatim}
b =
  103.5000
  175.8000
 -296.9000
 -450.1000
  197.4000
  656.6000
  412.4000
\end{verbatim}

\end{solution}
\end{computerExercise}


%%%%%%%%%%%%%%%%%%%%%%%%%%%%%%%%%%%%%%%%%%%%%%%%%%%%%%%%%%%%%%%%


\problemlabel

\exerciselabel{8}{3.2}\begin{exercise} \label{c4.2.2b}
Find a $2\times 2$ matrix that reflects vectors in the $(x,y)$ plane across
the $y$ axis.

\begin{solution}
\soln The map $L_A$ that reflects vectors across the $y$-axis is
$(x,y) \rightarrow (-x,y)$.  The matrix is
\[
A = \mattwo{-1}{0}{0}{1}.
\]

\end{solution}
\end{exercise}


%%%%%%%%%%%%%%%%%%%%%%%%%%%%%%%%%%%%%%%%%%%%%%%%%%%%%%%%%%%%%%%%



\problemlabel

\exerciselabel{1}{2.5}\begin{exercise} \label{c2.5.1}
Solve the system of equations
\[
\begin{array}{rcrcr}
 x_1 & - & ix_2 & = &  1\\
ix_1 & + & 3x_2 & = & -1
\end{array}
\]
Check your answer using \Matlabp.
\begin{prompt}
  $\left(\begin{array}{c} x_1 \\ x_2\end{array}\right) =
\left(\begin{array}{r} \answer{\frac{3}{2} - \frac{1}{2}i} \\ \answer{-\frac{1}{2} - \frac{1}{2}i}\end{array}\right).$
\end{prompt}

\begin{solution}

$\left(\begin{array}{c} x_1 \\ x_2\end{array}\right) =
\left(\begin{array}{r} \frac{3}{2} - \frac{1}{2}i \\ -\frac{1}{2} -
\frac{1}{2}i\end{array}\right).$

\end{solution}
\end{exercise}


%%%%%%%%%%%%%%%%%%%%%%%%%%%%%%%%%%%%%%%%%%%%%%%%%%%%%%%%%%%%%%%%



\problemlabel

\noindent In Exercises~\ref{c4.3.6a} -- \ref{c4.3.6d} determine
whether the given transformation is linear.


\exerciselabel{7}{3.3}\begin{exercise} \label{c4.3.6b}
  $T:\R^2\to\R^2$ defined by $T(x_1,x_2)=(x_1+x_1x_2,2x_2)$.

\begin{solution}
\ans The transformation $T(x,y) = (x + xy, 2y)$ is not linear.

\soln If $T$ is a linear transformation, then
\[
T(x_1 + x_2,y_1 + y_2) = T(x_1,y_1) + T(x_2,y_2)
\]
for any real numbers $x_1$,$x_2$,$y_1$,$y_2$.  However,
\[
\begin{array}{rcl}
T(1,1) & = & (2,2) \\
T(1,0) + T(0,1) & = & (1,0) + (0,2) = (1,2).\end{array}
\]
Therefore $T(1,1) \neq T(1,0) + T(0,1)$ and $T$ is not linear.

\end{solution}
\end{exercise}


%%%%%%%%%%%%%%%%%%%%%%%%%%%%%%%%%%%%%%%%%%%%%%%%%%%%%%%%%%%%%%%%



\problemlabel

\exerciselabel{9}{3.3}\begin{exercise} \label{c4.3.6d}
  $T:\R^2\to\R^3$ defined by $T(x_1,x_2)=(1,x_1+x_2,2x_2)$

\begin{solution}
The transformation $T(x,y) = (1,x + y, 2y)$ is not linear
because $T(0,0) = (1,0,0) \neq 0$.

\end{solution}
\end{exercise}


%%%%%%%%%%%%%%%%%%%%%%%%%%%%%%%%%%%%%%%%%%%%%%%%%%%%%%%%%%%%%%%%



\problemlabel

\exerciselabel{12}{3.3}\begin{exercise} \label{c4.3.8}
The {\em cross product\/} of two $3$-vectors $x=(x_1,x_2,x_3)$
and $y=(y_1,y_2,y_3)$ is the $3$-vector
\[
x\times y = (x_2y_3-x_3y_2,-(x_1y_3-x_3y_1),x_1y_2-x_2y_1).
\]
Let $K=(2,1,-1)$.  
\begin{enumeratea}
\item Show that the mapping $L:\R^3\to\R^3$ defined by
\[
L(x) = x\times K
\]
is a linear mapping.  
\item
Find the $3\times 3$ matrix $A$ such that
\[
L(x) = Ax,
\]
that is, $L=L_A$.
\end{enumeratea}
\begin{solution}

\ans The matrix of linear mapping $L$ is
\[
A = \matthree{0}{-1}{-1}{1}{0}{2}{1}{-2}{0}.
\]

\soln Let $X = (x_1,x_2,x_3)$ and let $Y = (y_1,y_2,y_3)$.  
Since $K = (2,1,-1)$,
\[
L(X) = (x_1,x_2,x_3) \times K = 
(-x_2 - x_3, x_1 + 2x_3, x_1 - 2x_2).
\]
\begin{enumeratea}
\item To show that $L(X)$ is a linear mapping, first demonstrate that
\eqref{sum} is valid:
\[
\begin{array}{rcl}
L(X + Y) & = & L(x_1 + y_1,x_2 + y_2,x_3 + y_3) \\
& = & (-(x_2 + y_2) - (x_3 + y_3), (x_1 + y_1) + 2(x_3 + y_3),
(x_1 + y_1) - 2(x_2 + y_2)) \\
& = & (-x_2 - x_3, x_1 + 2x_3, x_1 - 2x_2) +
(-y_2 - y_3, y_1 + 2y_3, y_1 - 2y_2) \\
& = & L(X) + L(Y), \end{array}
\]
then show that \eqref{product} is valid:
\[
\begin{array}{rcl}
cL(X) & = & cL(x_1,x_2,x_3) \\
& = & c(-x_2 - x_3, x_1 + 2x_3, x_1 - 2x_2) \\
& = & (-cx_2 - cx_3, cx_1 + 2cx_3, cx_1 - 2cx_2) \\
& = & L(cx_1,cx_2,cx_3) \\
& = & L(cX). \end{array}
\]

\item Find $A$ by noting that $L(e_j) = Ae_j$ is the $j^{th}$ column of $A$,
and computing
\[ \begin{array}{l}
L(e_1) = L(1,0,0) = (0,1,1) \\
L(e_2) = L(0,1,0) = (-1,0,-2) \\
L(e_3) = L(0,0,1) = (-1,2,0). \end{array} \]
\end{enumeratea}

\end{solution}
\end{exercise}


%%%%%%%%%%%%%%%%%%%%%%%%%%%%%%%%%%%%%%%%%%%%%%%%%%%%%%%%%%%%%%%%



\problemlabel

\exerciselabel{9}{3.1}\begin{exercise} \label{c4.1.4}
Write the system of linear equations
\begin{eqnarray*}
2x_1 + 3x_2 - 2x_3 & = & 4\\
6x_1 -5x_3 & = & 1
\end{eqnarray*}
in the matrix form $Ax=b$.

\begin{solution}

\[
\left(\begin{array}{rrr} 2 & 3 & -2 \\ 6 & 0 & -5\end{array}\right) 
\left(\begin{array}{r} x_1 \\ x_2 \\ x_3\end{array}\right) = 
\left(\begin{array}{r} 4 \\ 1\end{array}\right)
\]


\end{solution}
\end{exercise}


%%%%%%%%%%%%%%%%%%%%%%%%%%%%%%%%%%%%%%%%%%%%%%%%%%%%%%%%%%%%%%%%



\problemlabel

\exerciselabel{10}{3.1}\begin{exercise} \label{c4.1.6}
Find all solutions to
\[
\left(\begin{array}{rrrr} 1 & 3 & -1 & 4 \\ 2 & 1 & 5 & 7 \\
3 & 4 & 4 & 11 \end{array} \right)
\left(\begin{array}{c} x_1 \\ x_2 \\ x_3 \\ x_4\end{array}\right) =
\left(\begin{array}{c} 14 \\ 17 \\31 \end{array}\right).
\]

\begin{solution}

\ans All solutions are of the form
\[ \left(\begin{array}{r} x_1 \\ x_2 \\ x_3 \\ x_4\end{array}\right) =
\left(\begin{array}{c} \frac{37}{5} - \frac{16}{5}x_3 - \frac{17}{5}x_4 \\
\frac{11}{5} + \frac{7}{5}x_3 - \frac{1}{5}x_4 \\ x_3 \\ x_4\end{array}\right)
\]
where $x_3$ and $x_4$ are free parameters.

\soln Create the augmented matrix
\[ \left(\begin{array}{rrrr|r}
1 & 3 & -1 & 4 & 14 \\
2 & 1 & 5 & 7 & 17 \\
3 & 4 & 4 & 11 & 31\end{array}\right) \]
which can be row reduced to
\[ \left(\begin{array}{rrrr|r}
1 & 0 & \frac{16}{5} & \frac{17}{5} & \frac{37}{5} \\
0 & 1 & -\frac{7}{5} & \frac{1}{5} & \frac{11}{5} \\
0 & 0 & 0 & 0 & 0\end{array}\right), \]
yielding the desired solution.

\end{solution}
\end{exercise}


%%%%%%%%%%%%%%%%%%%%%%%%%%%%%%%%%%%%%%%%%%%%%%%%%%%%%%%%%%%%%%%%



\problemlabel

\noindent In Exercises~\ref{c4.2.a1a} -- \ref{c4.2.a1c} find a nonzero
vector that is mapped to the origin by the given matrix.


\exerciselabel{3}{3.2}\begin{exercise} \label{c4.2.a1c}
$C=\mattwo{3}{-1}{-6}{2}$.

\begin{solution}

\ans If $x = (x_1,3x_1)^t$, where $x_1$ is any real scalar, then $Cx = 0$.

\soln Solve $Cx = 0$ by row reducing $C$ to find that $Cx = 0$ when
$x_1 - \frac{1}{3}x_2 = 0$.

\end{solution}
\end{exercise}


%%%%%%%%%%%%%%%%%%%%%%%%%%%%%%%%%%%%%%%%%%%%%%%%%%%%%%%%%%%%%%%%



