\documentclass{article}

\input{../preamble}

\title{Homework 2}
\author{Dr. Golubitsky}
\date{Due: Wednesday, September 9, 2020 \@9:10am}

%\makeatletter
%\newlabel{c2.3.6a}{{1}{75}}
%\newlabel{c2.3.6c}{{3}{75}}
%\newlabel{c2.3.7a}{{4}{75}}
%\newlabel{c2.3.7c}{{6}{75}}
%\newlabel{e:refexamp6}{{2.3.14}{74}}
%\newlabel{c2.3.10a}{{10}{77}}
%\newlabel{c2.3.10b}{{11}{78}}
%\newlabel{c2.3.1a}{{21}{80}}
%\newlabel{c2.3.1c}{{23}{81}}
%\newlabel{number}{{2.4.6}{89}}
%\newlabel{c2.4.3a}{{7}{95}}
%\newlabel{c2.4.3d}{{10}{96}}
%\makeatother
\begin{document}
\maketitle


\problemlabel



\exerciselabel{9}{1.4}\begin{exercise} \label{c1.4.2}
Find a real number $a\begin{prompt}=\answer{10/3}\end{prompt}$ so that the vectors
\[
x = (1,3,2) \AND y = (2,a,-6)
\]
are perpendicular.
\begin{hint}
  The vectors $x$ and $y$ are perpendicular when
$(1,3,2) \cdot (2,a,-6) = 3a - 10 = 0$.
\end{hint}
\begin{hint}
  This means that $a = \frac{10}{3}$.
\end{hint}

\begin{solution}
\ans $a = \frac{10}{3}$.

\soln The vectors $x$ and $y$ are perpendicular when $(1,3,2) \cdot (2,a,-6) = 3a - 10 = 0$.  Thus, $a = \frac{10}{3}$.

\end{solution}
\end{exercise}


%%%%%%%%%%%%%%%%%%%%%%%%%%%%%%%%%%%%%%%%%%%%%%%%%%%%%%%%%%%%%%%%


\problemlabel

\exerciselabel{7}{2.1}\begin{exercise} \label{c2.1.11}
\begin{itemize}
\item[(a)] Find a quadratic polynomial $p(x) = ax^2 + bx + c$
  satisfying $p(0) = 1$, $p(1) = 5$, and $p(-1) = -5$.
  \begin{prompt}
    The quadratic $p(x) = \answer{-x^2 + 5x + 1}$ satisfies these conditions.
  \end{prompt}
  \begin{hint}
    Since $p(x) = ax^2 + bx + c$ for any quadratic equation, we find
this solution by evaluating $p(0) = 1$, $p(1) = 5$, and $p(-1) = -5$,
which yields the system of equations
\[
\begin{array}{lrrrrrrrr}
p(0) & = & & & & & c & = & 1 \\
p(1) & = & a & + & b & + & c & = & 5 \\
p(-1) & = & a & - & b & + & c & = & -5\end{array}
\]
We solve this system to obtain $(a,b,c) = (-1,5,1)$, then substitute
these coefficients into the general quadratic.
  \end{hint}
\item[(b)] Prove that for every triple of real numbers $L$, $M$,
and $N$, there is a quadratic polynomial satisfying $p(0) = L$,
$p(1) = M$, and $p(-1) = N$.
\begin{hint}
   Let $p(x) = ax^2 + bx + c$ be a quadratic equation.  Then, the
assumptions $p(0) = L$, $p(1) = M$, and $p(-1) = N$ imply:
\[
\begin{array}{lrrrrrrrr}
p(0) & = & & & & & c & = & L \\
p(1) & = & a & + & b & + & c & = & M \\
p(-1) & = & a & - & b & + & c & = & N\end{array}
\]
The unique solution to this system is $(a,b,c) =
(\frac{M + N - 2L}{2},\frac{M - N}{2},L)$.
\end{hint}
\item[(c)] Let $x_1,x_2,x_3$ be three unequal real
numbers and let $A_1,A_2,A_3$ be three real numbers.  Show
that finding a quadratic polynomial $q(x)$ that satisfies
$q(x_i) = A_i$ is equivalent to solving a system of three
linear equations.
\begin{hint}
   Substituting $q(x_i) = A_i$, for $i = 1,2,3$, into the standard
quadratic equation $q(x) = ax^2 + bx + c$ yields
\[
\begin{array}{ccccccc}
ax_1^2 & + & bx_1 & + & c & = & A_1 \\
ax_2^2 & + & bx_2 & + & c & = & A_2 \\
ax_3^2 & + & bx_3 & + & c & = & A_3\end{array}
\]
Finding the appropriate quadratic polynomial would be equivalent to
solving this system of linear equations for $a$, $b$, and $c$ in
terms of $A_1$, $A_2$, and $A_3$.
\end{hint}
\end{itemize}

\begin{solution}

(a) \ans The quadratic $p(x) = -x^2 + 5x + 1$ satisfies these conditions.

\soln Since $p(x) = ax^2 + bx + c$ for any quadratic equation, we find
this solution by evaluating $p(0) = 1$, $p(1) = 5$, and $p(-1) = -5$,
which yields the system of equations
\[
\begin{array}{lrrrrrrrr}
p(0) & = & & & & & c & = & 1 \\
p(1) & = & a & + & b & + & c & = & 5 \\
p(-1) & = & a & - & b & + & c & = & -5\end{array}
\]
We solve this system to obtain $(a,b,c) = (-1,5,1)$, then substitute
these coefficients into the general quadratic.

(b) Let $p(x) = ax^2 + bx + c$ be a quadratic equation.  Then, the
assumptions $p(0) = L$, $p(1) = M$, and $p(-1) = N$ imply:
\[
\begin{array}{lrrrrrrrr}
p(0) & = & & & & & c & = & L \\
p(1) & = & a & + & b & + & c & = & M \\
p(-1) & = & a & - & b & + & c & = & N\end{array}
\]
The unique solution to this system is $(a,b,c) =
(\frac{M + N - 2L}{2},\frac{M - N}{2},L)$.

(c) Substituting $q(x_i) = A_i$, for $i = 1,2,3$, into the standard
quadratic equation $q(x) = ax^2 + bx + c$ yields
\[
\begin{array}{ccccccc}
ax_1^2 & + & bx_1 & + & c & = & A_1 \\
ax_2^2 & + & bx_2 & + & c & = & A_2 \\
ax_3^2 & + & bx_3 & + & c & = & A_3\end{array}
\]
Finding the appropriate quadratic polynomial would be equivalent to
solving this system of linear equations for $a$, $b$, and $c$ in
terms of $A_1$, $A_2$, and $A_3$.




\end{solution}
\end{exercise}


%%%%%%%%%%%%%%%%%%%%%%%%%%%%%%%%%%%%%%%%%%%%%%%%%%%%%%%%%%%%%%%%


\problemlabel

% To which section does your exercise belong? 


\exerciselabel{5}{2.2}\begin{exercise}\label{c2.2.85}

Find the cosine of the angle between the normal vectors to the planes 
\[
2x - 2y + z = 14 \AND x + y - 2z = -10.
\]

  
\begin{solution}

\ans $-\frac{2}{3\sqrt{6}}$

\soln The normal vectors are $v = (2, -2, 1)$ and $w = ( 1, 1 , -2)$.  The cosine of the angle $\theta$ between the normal vectors is 
\[
\cos(\theta) = \frac{v\cdot w}{||v||\; ||w||} = \frac{-2}{\sqrt{9}\sqrt{6}} = -\frac{2}{3\sqrt{6}}
\]

\end{solution}
\end{exercise}


%%%%%%%%%%%%%%%%%%%%%%%%%%%%%%%%%%%%%%%%%%%%%%%%%%%%%%%%%%%%%%%%


\problemlabel

\noindent In Exercises~\ref{c2.3.6a} -- \ref{c2.3.6c} determine
whether the given matrix is in reduced echelon form.


\exerciselabel{2}{2.3}\begin{exercise} \label{c2.3.6b}
$\left(\begin{array}{rrrr}
1 &  0 & -2 &   0   \\
0 &  1 &  4 &   0    \\
         0 &  0 &  0 &   1  \end{array}\right)$.

\begin{solution}
\ans The matrix is in reduced echelon form.

\end{solution}
\end{exercise}


%%%%%%%%%%%%%%%%%%%%%%%%%%%%%%%%%%%%%%%%%%%%%%%%%%%%%%%%%%%%%%%%


\problemlabel

\noindent In Exercises~\ref{c2.3.7a} -- \ref{c2.3.7c} we list
the reduced echelon form of an augmented matrix of a system of
linear equations.  Which columns in these augmented matrices
contain pivots?  Describe all solutions to these systems of
equations in the form of \eqref{e:refexamp6}.


\exerciselabel{5}{2.3}\begin{exercise} \label{c2.3.7b}
$\left(\begin{array}{rrrr|r}
 1  &  2 & 0 & 0 & 0\\
 0  &  0 & 1 & 1 & 0\\
 0  &  0 & 0 & 0 & 1
       \end{array}\right)$.

\begin{solution}
\ans The $1^{st}$, $3^{rd}$, and $5^{th}$ columns of the matrix
contain pivots.  Since the last row of the matrix translates to the linear
equation $0 = 1$, the system is inconsistent, and there are no solutions.

\end{solution}
\end{exercise}


%%%%%%%%%%%%%%%%%%%%%%%%%%%%%%%%%%%%%%%%%%%%%%%%%%%%%%%%%%%%%%%%


\problemlabel

\exerciselabel{9}{2.3}\begin{exercise} \label{c2.3.9}
Use row reduction and back substitution to solve the following
system of two equations in three unknowns:
\[
\begin{array}{rcrcrcrc}
 x_1 & - & x_2 & + & x_3 & = & 1 \\
2x_1 & + & x_2 & - & x_3 & = & -1
\end{array}
\]

\begin{solution}

\ans The solution to this system is
\[
\left(\begin{array}{c} x_1 \\ x_2 \\ x_3\end{array}\right) =
\left(\begin{array}{c} 0 \\ x_3 - 1 \\ x_3\end{array}\right) = \Matrix{0 \\ -1 \\ 0} + x_3\Matrix{0 \\ 1 \\ 1},
\]
where $x_3$ is any real number.

\soln Row reduce the augmented matrix of the system:
\[
\left(\begin{array}{rrr|r} 1 & -1 & 1 & 1 \\ 2 & 1 & -1 & -1\end{array}\right)
\longrightarrow
\left(\begin{array}{rrr|r} 1 & 0 & 0 & 0 \\ 0 & 1 & -1 & -1\end{array}\right).
\]
Although $(1,2,2)$ is not a solution to this system, there is a solution
for which $x_3 = 2$, namely $(0,1,2)$.

\end{solution}
\end{exercise}


%%%%%%%%%%%%%%%%%%%%%%%%%%%%%%%%%%%%%%%%%%%%%%%%%%%%%%%%%%%%%%%%


\problemlabel

\noindent In Exercises~\ref{c2.3.10a} -- \ref{c2.3.10b} determine the
augmented matrix and all solutions for each system of linear equations


\exerciselabel{11}{2.3}\begin{exercise} \label{c2.3.10b}
$\begin{array}{rcl}
2x-y+z+w & = & 1\\
   x+2y-z+w & = & 7 \end{array}$.


\begin{solution}
\soln The augmented matrix for this system is
\[
\left(\begin{array}{rrrr|r} 2 & -1 & 1 & 1 & 1 \\ 1 & 2 & -1 & 1 & 7
\end{array}\right)
\]
which can be row reduced to
\[
\left(\begin{array}{rrrr|r} 1 & 0 & 1/5 & 3/5 &
9/5 \\ 0 & 1 & -3/5 & 1/5 & 13/5
\end{array}\right).
\]
The solution set is therefore
\[
\left(\begin{array}{c} x \\ y \\ z \\ w \end{array}\right) =
\left(\begin{array}{c} 9/5 - 1/5\; z - 3/5\; w
\\ 13/5 + 3/5\; z - 1/5 \; w \\ z \\ w \end{array}\right)=
\Matrixc{9/5 \\ 13/5 \\ 0 \\ 0} + z\Matrixc{-1/5\\  3/5 \\ 1 \\ 0} + w\Matrixc{-3/5 \\ -1/5 \\ 0 \\ 1}.
\]

\end{solution}
\end{exercise}


%%%%%%%%%%%%%%%%%%%%%%%%%%%%%%%%%%%%%%%%%%%%%%%%%%%%%%%%%%%%%%%%


\matlabproblemlabel

\noindent In Exercises~\ref{c2.3.1a} -- \ref{c2.3.1c} use elementary row
operations and \Matlab to put each of the given matrices into row echelon
form.  Suppose that the matrix is the augmented matrix for a system of
linear equations.  Is the system consistent or inconsistent?


\exerciselabel{23}{2.3}\begin{computerExercise} \label{c2.3.1c}
\[
\left(\begin{array}{rrrr}
 -2 & 1 &  9 & 1\\
  3 & 3 & -4 & 2\\
  1 & 4 &  5 & 5
\end{array}\right).
\]

\begin{solution}
\ans The system is inconsistent.

\soln The row-reduced matrix is:
A = 
\begin{verbatim}
1.0000         0   -3.4444         0
         0    1.0000    2.1111         0
         0         0         0    1.0000
\end{verbatim}
This matrix represents an inconsistent linear system.


\end{solution}
\end{computerExercise}


%%%%%%%%%%%%%%%%%%%%%%%%%%%%%%%%%%%%%%%%%%%%%%%%%%%%%%%%%%%%%%%%


\problemlabel

\exerciselabel{4}{2.4}\begin{exercise} \label{c2.4.2}
The augmented matrix of a consistent system of five equations in seven
unknowns has rank equal to three.  How many parameters are needed to
specify all solutions?
\begin{prompt}
  There are $\answer{4}$ parameters needed to specify all solutions.
\end{prompt}
\begin{hint}
  According to Theorem \ref{number}, $n - \ell$
parameters are needed to parameterize the set of all solutions of a
linear system, where $n$ is the number of unknowns, and $\ell$ is the
rank of the reduced echelon matrix.  In this case, $n = 7$ and $\ell =
3$.
\end{hint}

\begin{solution}

\ans Four parameters are needed to specify all solutions.

\soln According to Theorem \ref{number}, $n - \ell$
parameters are needed to parameterize the set of all solutions of a
linear system, where $n$ is the number of unknowns, and $\ell$ is the
rank of the reduced echelon matrix.  In this case, $n = 7$ and $\ell =
3$.

\end{solution}
\end{exercise}


%%%%%%%%%%%%%%%%%%%%%%%%%%%%%%%%%%%%%%%%%%%%%%%%%%%%%%%%%%%%%%%%


\matlabproblemlabel

\noindent In Exercises~\ref{c2.4.3a} -- \ref{c2.4.3d}, use {\tt rref} on
the given augmented matrices to determine whether the associated system of
linear equations is consistent or inconsistent.  If the equations are
consistent, then determine how many parameters are needed to enumerate
all solutions.


\exerciselabel{7}{2.4}\begin{computerExercise} \label{c2.4.3a}
\begin{matlabEquation}\label{MATLAB:17}
A = \left(\begin{array}{rrrrr|r}
2 & 1 & 3 & -2 & 4 & 1 \\
5 & 12 & -1 & 3 & 5 & 1 \\
-4  &  -21 &    11  &  -12  &    2  &    1  \\
23  &  59  &  -8   & 17  &  21  &   4
\end{array}\right) \quad
\end{matlabEquation}

\begin{solution}

\ans Matrix $A$ is consistent and requires 3 parameters to enumerate
all solutions.

\soln
\begin{verbatim}
rref(A) = 
    1.0000         0    1.9474   -1.4211    2.2632    0.5789
         0    1.0000   -0.8947    0.8421   -0.5263   -0.1579
         0         0         0         0         0         0
         0         0         0         0         0         0
\end{verbatim}

\end{solution}
\end{computerExercise}


%%%%%%%%%%%%%%%%%%%%%%%%%%%%%%%%%%%%%%%%%%%%%%%%%%%%%%%%%%%%%%%%


\end{document}
