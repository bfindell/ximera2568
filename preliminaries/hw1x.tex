\documentclass{ximera}
%\def\MakeUppercaseUnsupportedInPdfStrings{\scshape}

\input{../preamble}

\title{Math 2568 Homework 0}
\date{Due: September 2, 2022}
\author{Martin Golubitsky, Michael Dellnitz, and Brad Findell}

\makeatletter
%\newlabel{c1.1.1A}{{1}{3}}
%\newlabel{c1.1.1C}{{3}{4}}
%\newlabel{c1.1.3a}{{5}{4}}
%\newlabel{c1.1.3e}{{9}{4}}
%\newlabel{c1.1.4A}{{10}{5}}
%\newlabel{c1.1.4B}{{11}{5}}
\makeatother
\begin{document}
\begin{abstract}
Online portions of Homework 0.
\end{abstract}
\maketitle


\problemlabel

\noindent In Exercises~\ref{c1.1.1A} -- \ref{c1.1.1C}, let $x=(2,1,3)$ and 
$y=(1,1,-1)$ and compute the given expression.


\exerciselabel{1}{1.1}\begin{exercise}  \label{c1.1.1A}
  $x+y\begin{prompt}
    = \left(\answer{3},\answer{2},\answer{2}\right).
  \end{prompt}$.

\begin{solution}
\ans $ x + y = (3,2,2)$.

\end{solution}
\end{exercise}


%%%%%%%%%%%%%%%%%%%%%%%%%%%%%%%%%%%%%%%%%%%%%%%%%%%%%%%%%%%%%%%%


\problemlabel

\exerciselabel{3}{1.1}\begin{exercise}  \label{c1.1.1C}
  $4x\begin{prompt}
    = \left(\answer{8},\answer{4},\answer{12}\right)
    \end{prompt}$.

\begin{solution}
\ans $4x = (8,4,12)$.
\end{solution}
\end{exercise}


%%%%%%%%%%%%%%%%%%%%%%%%%%%%%%%%%%%%%%%%%%%%%%%%%%%%%%%%%%%%%%%%


\problemlabel



\exerciselabel{4}{1.1}\begin{exercise} \label{c1.1.2}
Let $A$ be the $3\times 4$ matrix
\[
A=\left(\begin{array}{rrrr} 2 & -1 & 0 & 1 \\ 3 & 4 & -7 & 10\\
        6 & -3 & 4 & 2 \end{array}\right).
\]
\begin{enumerate}
\item[(a)]  For which $n$ is a row of $A$ a vector in $\R^n$? \begin{prompt}\[n = \answer{4}\]\end{prompt}.
\item[(b)]  What is the $2^{nd}$ column of $A$?
  \begin{prompt}
    \[
      \left(\begin{array}{r} \answer{-1} \\ \answer{4} \\ \answer{-3} \end{array} \right)
    \]
  \end{prompt}
\item[(c)] Let $a_{ij}$ be the entry of $A$ in the $i^{th}$ row
  and the $j^{th}$ column.  What is $a_{23}-a_{31}$?
  \begin{prompt}
    \[
      a_{23}-a_{31} = \answer{-13}.
    \]
  \end{prompt}
\end{enumerate}

\begin{solution}
\ans
(a) The number of entries in a row is the number of columns.  Thus, $n = 4$; \\
(b) $\left(\begin{array}{r} -1 \\ 4 \\ -3 \end{array} \right)$;
(c) $a_{23}-a_{31} =  -7 - 6 = -13$.

\end{solution}
\end{exercise}


%%%%%%%%%%%%%%%%%%%%%%%%%%%%%%%%%%%%%%%%%%%%%%%%%%%%%%%%%%%%%%%%


\problemlabel

\noindent For each of the pairs of vectors or matrices in
Exercises~\ref{c1.1.3a} -- \ref{c1.1.3e}, decide whether addition
of the members of the pair is possible; and, if addition is possible,
perform the addition.


\exerciselabel{5}{1.1}\begin{exercise}\label{c1.1.3a}
  $x=(2,1)$ and $y=(3,-1)$.
  
\begin{solution}
\ans $x + y = (5,0)$.

\end{solution}
\end{exercise}


%%%%%%%%%%%%%%%%%%%%%%%%%%%%%%%%%%%%%%%%%%%%%%%%%%%%%%%%%%%%%%%%


\problemlabel

\exerciselabel{7}{1.1}\begin{exercise}\label{c1.1.3c}
  $x=(1,2,3)$ and $y=(-2,1)$.
  

\begin{solution}
\ans $x$ has three entries; $y$ has two entries; addition is not possible.

\end{solution}
\end{exercise}


%%%%%%%%%%%%%%%%%%%%%%%%%%%%%%%%%%%%%%%%%%%%%%%%%%%%%%%%%%%%%%%%


\problemlabel

\exerciselabel{8}{1.1}\begin{exercise}\label{c1.1.3d}
  $A=\mattwo{1}{3}{0}{4}$ and $B=\mattwo{2}{1}{1}{-2}$.
  
\begin{solution}
\ans $A + B = \mattwo{3}{4}{1}{2}$.

\end{solution}

\begin{solution}


\end{solution}
\end{exercise}


%%%%%%%%%%%%%%%%%%%%%%%%%%%%%%%%%%%%%%%%%%%%%%%%%%%%%%%%%%%%%%%%


\problemlabel

\noindent In Exercises~\ref{c1.1.4A} -- \ref{c1.1.4B}, let
$A=\mattwo{2}{1}{-1}{4}$ and $B=\mattwo{0}{2}{3}{-1}$ and compute the given 
expression.


\exerciselabel{11}{1.1}\begin{exercise}\label{c1.1.4B}
  $2A-3B\begin{prompt}=\mattwo{\answer{4}}{\answer{-4}}{\answer{-11}}{\answer{11}}\end{prompt}$.

\begin{solution}
\ans $2A - 3B = \mattwo{4}{-4}{-11}{11}$.




\end{solution}
\end{exercise}


%%%%%%%%%%%%%%%%%%%%%%%%%%%%%%%%%%%%%%%%%%%%%%%%%%%%%%%%%%%%%%%%



\end{document}
