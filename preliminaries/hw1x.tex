\documentclass{ximera}
%\def\MakeUppercaseUnsupportedInPdfStrings{\scshape}

\usepackage{pgf,tikz}
\usepackage{mathrsfs}
\usetikzlibrary{shapes,arrows}
\usepackage{framed}
\pgfplotsset{compat=1.13}

\graphicspath{
  {./}
  {ximeraTutorial/}
}

\newenvironment{sectionOutcomes}{}{}


% Below is the preamble from the textbook in /laode, by Golubitsky and Dellnitz
% ... with problematic stuff commented out
% Followed by the hw-preamble from worksheet-builder

%\usepackage{ulem}
\usepackage[normalem]{ulem}

\epstopdfsetup{outdir=./}

\usepackage{morewrites}
\makeatletter
\newcommand\subfile[1]{%
\renewcommand{\input}[1]{}%
\begingroup\skip@preamble\otherinput{#1}\endgroup\par\vspace{\topsep}
\let\input\otherinput}
\makeatother

\newcommand{\EXER}{}
\newcommand{\includeexercises}{\EXER\directlua{dofile(kpse.find_file("exercises","lua"))}}

\newenvironment{computerExercise}{\begin{exercise}}{\end{exercise}}

%\newcounter{ccounter}
%\setcounter{ccounter}{1}
%\newcommand{\Chapter}[1]{\setcounter{chapter}{\arabic{ccounter}}\chapter{#1}\addtocounter{ccounter}{1}}

%\newcommand{\section}[1]{\section{#1}\setcounter{thm}{0}\setcounter{equation}{0}}

%\renewcommand{\theequation}{\arabic{chapter}.\arabic{section}.\arabic{equation}}
%\renewcommand{\thefigure}{\arabic{chapter}.\arabic{figure}}
%\renewcommand{\thetable}{\arabic{chapter}.\arabic{table}}

%\newcommand{\Sec}[2]{\section{#1}\markright{\arabic{ccounter}.\arabic{section}.#2}\setcounter{equation}{0}\setcounter{thm}{0}\setcounter{figure}{0}}
  
\newcommand{\Sec}[2]{\section{#1}}

\setcounter{secnumdepth}{2}
%\setcounter{secnumdepth}{1} 

%\newcounter{THM}
%\renewcommand{\theTHM}{\arabic{chapter}.\arabic{section}}

\newcommand{\trademark}{{R\!\!\!\!\!\bigcirc}}
%\newtheorem{exercise}{}

\newcommand{\dfield}{{\sf SlopeField}}

\newcommand{\pplane}{{\sf PhasePlane}}

\newcommand{\PPLANE}{{\sf PHASEPLANE}}

% BADBAD: \newcommand{\Bbb}{\bf}. % Package amsfonts Warning: Obsolete command \Bbb; \mathbb should be used instead.

\newcommand{\R}{\mbox{$\mathbb{R}$}}
\let\C\relax
\newcommand{\C}{\mbox{$\mathbb{C}$}}
\newcommand{\Z}{\mbox{$\mathbb{Z}$}}
\newcommand{\N}{\mbox{$\mathbb{N}$}}
\newcommand{\D}{\mbox{{\bf D}}}

\newcommand{\WW}{\mathcal{W}}

\usepackage{amssymb}
%\newcommand{\qed}{\hfill\mbox{\raggedright$\square$} \vspace{1ex}}
%\newcommand{\proof}{\noindent {\bf Proof:} \hspace{0.1in}}

\newcommand{\setmin}{\;\mbox{--}\;}
\newcommand{\Matlab}{{M\small{AT\-LAB}} }
\newcommand{\Matlabp}{{M\small{AT\-LAB}}}
\newcommand{\computer}{\Matlab Instructions}
\renewcommand{\computer}{M\small{ATLAB} Instructions}
\newcommand{\half}{\mbox{$\frac{1}{2}$}}
\newcommand{\compose}{\raisebox{.15ex}{\mbox{{\scriptsize$\circ$}}}}
\newcommand{\AND}{\quad\mbox{and}\quad}
\newcommand{\vect}[2]{\left(\begin{array}{c} #1_1 \\ \vdots \\
 #1_{#2}\end{array}\right)}
\newcommand{\mattwo}[4]{\left(\begin{array}{rr} #1 & #2\\ #3
&#4\end{array}\right)}
\newcommand{\mattwoc}[4]{\left(\begin{array}{cc} #1 & #2\\ #3
&#4\end{array}\right)}
\newcommand{\vectwo}[2]{\left(\begin{array}{r} #1 \\ #2\end{array}\right)}
\newcommand{\vectwoc}[2]{\left(\begin{array}{c} #1 \\ #2\end{array}\right)}

\newcommand{\ignore}[1]{}


\newcommand{\inv}{^{-1}}
\newcommand{\CC}{{\cal C}}
\newcommand{\CCone}{\CC^1}
\newcommand{\Span}{{\rm span}}
\newcommand{\rank}{{\rm rank}}
\newcommand{\trace}{{\rm tr}}
\newcommand{\RE}{{\rm Re}}
\newcommand{\IM}{{\rm Im}}
\newcommand{\nulls}{{\rm null\;space}}

\newcommand{\dps}{\displaystyle}
\newcommand{\arraystart}{\renewcommand{\arraystretch}{1.8}}
\newcommand{\arrayfinish}{\renewcommand{\arraystretch}{1.2}}
\newcommand{\Start}[1]{\vspace{0.08in}\noindent {\bf Section~\ref{#1}}}
\newcommand{\exer}[1]{\noindent {\bf \ref{#1}}}
\newcommand{\ans}{\textbf{Answer:} }
\newcommand{\matthree}[9]{\left(\begin{array}{rrr} #1 & #2 & #3 \\ #4 & #5 & #6
\\ #7 & #8 & #9\end{array}\right)}
\newcommand{\cvectwo}[2]{\left(\begin{array}{c} #1 \\ #2\end{array}\right)}
\newcommand{\cmatthree}[9]{\left(\begin{array}{ccc} #1 & #2 & #3 \\ #4 & #5 &
#6 \\ #7 & #8 & #9\end{array}\right)}
\newcommand{\vecthree}[3]{\left(\begin{array}{r} #1 \\ #2 \\
#3\end{array}\right)}
\newcommand{\cvecthree}[3]{\left(\begin{array}{c} #1 \\ #2 \\
#3\end{array}\right)}
\newcommand{\cmattwo}[4]{\left(\begin{array}{cc} #1 & #2\\ #3
&#4\end{array}\right)}

\newcommand{\Matrix}[1]{\ensuremath{\left(\begin{array}{rrrrrrrrrrrrrrrrrr} #1 \end{array}\right)}}

\newcommand{\Matrixc}[1]{\ensuremath{\left(\begin{array}{cccccccccccc} #1 \end{array}\right)}}



\renewcommand{\labelenumi}{\theenumi}
\newenvironment{enumeratea}%
{\begingroup
 \renewcommand{\theenumi}{\alph{enumi}}
 \renewcommand{\labelenumi}{(\theenumi)}
 \begin{enumerate}}
 {\end{enumerate}
 \endgroup}

\newcounter{help}
\renewcommand{\thehelp}{\thesection.\arabic{equation}}

%\newenvironment{equation*}%
%{\renewcommand\endequation{\eqno (\theequation)* $$}%
%   \begin{equation}}%
%   {\end{equation}\renewcommand\endequation{\eqno \@eqnnum
%$$\global\@ignoretrue}}

\author{Martin Golubitsky and Michael Dellnitz}

%\newenvironment{matlabEquation}%
%{\renewcommand\endequation{\eqno (\theequation*) $$}%
%   \begin{equation}}%
%   {\end{equation}\renewcommand\endequation{\eqno \@eqnnum
% $$\global\@ignoretrue}}

\newcommand{\soln}{\textbf{Solution:} }
\newcommand{\exercap}[1]{\centerline{Figure~\ref{#1}}}
\newcommand{\exercaptwo}[1]{\centerline{Figure~\ref{#1}a\hspace{2.1in}
Figure~\ref{#1}b}}
\newcommand{\exercapthree}[1]{\centerline{Figure~\ref{#1}a\hspace{1.2in}
Figure~\ref{#1}b\hspace{1.2in}Figure~\ref{#1}c}}
\newcommand{\para}{\hspace{0.4in}}

%\usepackage{ifluatex}
%\ifluatex
%\ifcsname displaysolutions\endcsname%
%\else
%\renewenvironment{solution}{\suppress}{\endsuppress}
%\fi
%\else
%\renewenvironment{solution}{}{}
%\fi
%
%\ifcsname answer\endcsname
%\renewcommand{\answer}{}
%\fi

%\ifxake
%\newenvironment{matlabEquation}{\begin{equation}}{\end{equation}}
%\else
\newenvironment{matlabEquation}%
{\let\oldtheequation\theequation\renewcommand{\theequation}{\oldtheequation*}\begin{equation}}%
  {\end{equation}\let\theequation\oldtheequation}
%\fi

%\makeatother

\newcommand{\RED}[1]{{\color{red}{#1}}} 


%%
%%
%% Worksheet-builder preamble
%%
%%

\usepackage{xcolor}
\renewenvironment{solution}{\color{blue}}{\color{black}}
\renewenvironment{computerExercise}{\begin{exercise}\textsc{(matlab)} }{\end{exercise}}

%\usepackage{environ}
%\RenewEnviron{prompt}{}
%\RenewEnviron{hint}{}
%\RenewEnviron{multipleChoice}{}
%\RenewEnviron{feedback}{}

\renewcommand{\ans}{\noindent\textbf{Answer: }}
\renewcommand{\soln}{\noindent\textbf{Solution: }}

%\renewcommand{\answer}[2][]{#2}

% if you want to hide solutions, uncomment the following
%\usepackage{comment}\excludecomment{solution}

\def\isitmatlab{}
\newcommand{\matlab}{\def\isitmatlab{ (MATLAB)}}

\makeatletter
\newcommand{\exerciselabel}[2]{\textbf{\textsection #2, Exercise #1\isitmatlab.}\def\@currentlabel{#1}\def\isitmatlab{}}
\makeatother

\newcounter{problemx}
\newcommand{\problemlabel}{\refstepcounter{problemx}\section*{Problem \arabic{problemx}}}
\newcommand{\matlabproblemlabel}{\refstepcounter{problemx}\section*{Problem \arabic{problemx} (MATLAB)}}
%\newcommand{\problemlabel}{\refstepcounter{problem}\section*{Problem}}
%\newcommand{\matlabproblemlabel}{\refstepcounter{problem}\section*{Problem (MATLAB)}}



\title{Math 2568 Homework 0}
\date{Due: September 2, 2022}
\author{Martin Golubitsky, Michael Dellnitz, and Brad Findell}

\makeatletter
%\newlabel{c1.1.1A}{{1}{3}}
%\newlabel{c1.1.1C}{{3}{4}}
%\newlabel{c1.1.3a}{{5}{4}}
%\newlabel{c1.1.3e}{{9}{4}}
%\newlabel{c1.1.4A}{{10}{5}}
%\newlabel{c1.1.4B}{{11}{5}}
\makeatother
\begin{document}
\begin{abstract}
Online portions of Homework 0.
\end{abstract}
\maketitle


\problemlabel

\noindent In Exercises~\ref{c1.1.1A} -- \ref{c1.1.1C}, let $x=(2,1,3)$ and 
$y=(1,1,-1)$ and compute the given expression.


\exerciselabel{1}{1.1}\begin{exercise}  \label{c1.1.1A}
  $x+y\begin{prompt}
    = \left(\answer{3},\answer{2},\answer{2}\right).
  \end{prompt}$.

\begin{solution}
\ans $ x + y = (3,2,2)$.

\end{solution}
\end{exercise}


%%%%%%%%%%%%%%%%%%%%%%%%%%%%%%%%%%%%%%%%%%%%%%%%%%%%%%%%%%%%%%%%


\problemlabel

\exerciselabel{3}{1.1}\begin{exercise}  \label{c1.1.1C}
  $4x\begin{prompt}
    = \left(\answer{8},\answer{4},\answer{12}\right)
    \end{prompt}$.

\begin{solution}
\ans $4x = (8,4,12)$.
\end{solution}
\end{exercise}


%%%%%%%%%%%%%%%%%%%%%%%%%%%%%%%%%%%%%%%%%%%%%%%%%%%%%%%%%%%%%%%%


\problemlabel



\exerciselabel{4}{1.1}\begin{exercise} \label{c1.1.2}
Let $A$ be the $3\times 4$ matrix
\[
A=\left(\begin{array}{rrrr} 2 & -1 & 0 & 1 \\ 3 & 4 & -7 & 10\\
        6 & -3 & 4 & 2 \end{array}\right).
\]
\begin{enumerate}
\item[(a)]  For which $n$ is a row of $A$ a vector in $\R^n$? \begin{prompt}\[n = \answer{4}\]\end{prompt}.
\item[(b)]  What is the $2^{nd}$ column of $A$?
  \begin{prompt}
    \[
      \left(\begin{array}{r} \answer{-1} \\ \answer{4} \\ \answer{-3} \end{array} \right)
    \]
  \end{prompt}
\item[(c)] Let $a_{ij}$ be the entry of $A$ in the $i^{th}$ row
  and the $j^{th}$ column.  What is $a_{23}-a_{31}$?
  \begin{prompt}
    \[
      a_{23}-a_{31} = \answer{-13}.
    \]
  \end{prompt}
\end{enumerate}

\begin{solution}
\ans
(a) The number of entries in a row is the number of columns.  Thus, $n = 4$; \\
(b) $\left(\begin{array}{r} -1 \\ 4 \\ -3 \end{array} \right)$;
(c) $a_{23}-a_{31} =  -7 - 6 = -13$.

\end{solution}
\end{exercise}


%%%%%%%%%%%%%%%%%%%%%%%%%%%%%%%%%%%%%%%%%%%%%%%%%%%%%%%%%%%%%%%%


\problemlabel

\noindent For each of the pairs of vectors or matrices in
Exercises~\ref{c1.1.3a} -- \ref{c1.1.3e}, decide whether addition
of the members of the pair is possible; and, if addition is possible,
perform the addition.


\exerciselabel{5}{1.1}\begin{exercise}\label{c1.1.3a}
  $x=(2,1)$ and $y=(3,-1)$.
  
\begin{solution}
\ans $x + y = (5,0)$.

\end{solution}
\end{exercise}


%%%%%%%%%%%%%%%%%%%%%%%%%%%%%%%%%%%%%%%%%%%%%%%%%%%%%%%%%%%%%%%%


\problemlabel

\exerciselabel{7}{1.1}\begin{exercise}\label{c1.1.3c}
  $x=(1,2,3)$ and $y=(-2,1)$.
  

\begin{solution}
\ans $x$ has three entries; $y$ has two entries; addition is not possible.

\end{solution}
\end{exercise}


%%%%%%%%%%%%%%%%%%%%%%%%%%%%%%%%%%%%%%%%%%%%%%%%%%%%%%%%%%%%%%%%


\problemlabel

\exerciselabel{8}{1.1}\begin{exercise}\label{c1.1.3d}
  $A=\mattwo{1}{3}{0}{4}$ and $B=\mattwo{2}{1}{1}{-2}$.
  
\begin{solution}
\ans $A + B = \mattwo{3}{4}{1}{2}$.

\end{solution}

\begin{solution}


\end{solution}
\end{exercise}


%%%%%%%%%%%%%%%%%%%%%%%%%%%%%%%%%%%%%%%%%%%%%%%%%%%%%%%%%%%%%%%%


\problemlabel

\noindent In Exercises~\ref{c1.1.4A} -- \ref{c1.1.4B}, let
$A=\mattwo{2}{1}{-1}{4}$ and $B=\mattwo{0}{2}{3}{-1}$ and compute the given 
expression.


\exerciselabel{11}{1.1}\begin{exercise}\label{c1.1.4B}
  $2A-3B\begin{prompt}=\mattwo{\answer{4}}{\answer{-4}}{\answer{-11}}{\answer{11}}\end{prompt}$.

\begin{solution}
\ans $2A - 3B = \mattwo{4}{-4}{-11}{11}$.




\end{solution}
\end{exercise}


%%%%%%%%%%%%%%%%%%%%%%%%%%%%%%%%%%%%%%%%%%%%%%%%%%%%%%%%%%%%%%%%



\end{document}
