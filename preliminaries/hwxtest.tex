\documentclass{ximera}
\def\MakeUppercaseUnsupportedInPdfStrings{\scshape}

\newcommand{\R}{{\mathbb R}}

\newcommand{\ans}{\textbf{Answer: } }
\newcommand{\soln}{\textbf{Solution: } }

\usepackage{xcolor}
\renewenvironment{solution}{\color{blue}}{\color{black}}
\newcommand{\exerciselabel}[2]{\textbf{\textsection #2, Exercise #1}\def\@currentlabel{#1}}
\newcommand{\problemlabel}{\refstepcounter{problem}\section*{Problem}}


\title{Math 2568 Homework 0}
\date{Due: September 2, 2022}
\author{Martin Golubitsky and Michael Dellnitz}

\makeatletter
%\newlabel{c1.1.1A}{{1}{3}}
%\newlabel{c1.1.1C}{{3}{4}}
\makeatother

\begin{document}
\begin{abstract}
Online portions of Homework 0.
\end{abstract}
\maketitle


\problemlabel

% Comment out the references in the line below, and the file compiles.  

\noindent In Exercises ~\ref{c1.1.1A} -- \ref{c1.1.1C}, 
let $x=(2,1,3)$ and 
$y=(1,1,-1)$ and compute the given expression.


\exerciselabel{1}{1.1}\begin{exercise}  \label{c1.1.1A}
  $x+y\begin{prompt}
    = \left(\answer{3},\answer{2},\answer{2}\right).
  \end{prompt}$.

\begin{solution}
\ans $ x + y = (3,2,2)$.

\end{solution}
\end{exercise}


%%%%%%%%%%%%%%%%%%%%%%%%%%%%%%%%%%%%%%%%%%%%%%%%%%%%%%%%%%%%%%%%


\problemlabel

\exerciselabel{2}{1.1}\begin{exercise}  \label{c1.1.1B}
  $2x-3y\begin{prompt}
    = \left(\answer{1},\answer{-1},\answer{9}\right)
  \end{prompt}$.
  \begin{hint}
    $2x - 3y = (4,2,6) - (3,3,-3)$.
  \end{hint}
  \begin{hint}
    $(4,2,6) - (3,3,-3) = (1,-1,9)$.
  \end{hint}  

\begin{solution}
\ans $2x - 3y = (4,2,6) - (3,3,-3) = (1,-1,9)$.

\end{solution}
\end{exercise}


%%%%%%%%%%%%%%%%%%%%%%%%%%%%%%%%%%%%%%%%%%%%%%%%%%%%%%%%%%%%%%%%


\problemlabel

\exerciselabel{3}{1.1}\begin{exercise}  \label{c1.1.1C}
  $4x\begin{prompt}
    = \left(\answer{8},\answer{4},\answer{12}\right)
    \end{prompt}$.

\begin{solution}
\ans $4x = (8,4,12)$.
\end{solution}
\end{exercise}


%%%%%%%%%%%%%%%%%%%%%%%%%%%%%%%%%%%%%%%%%%%%%%%%%%%%%%%%%%%%%%%%


\problemlabel



\exerciselabel{4}{1.1}\begin{exercise} \label{c1.1.2}
Let $A$ be the $3\times 4$ matrix
\[
A=\left(\begin{array}{rrrr} 2 & -1 & 0 & 1 \\ 3 & 4 & -7 & 10\\
        6 & -3 & 4 & 2 \end{array}\right).
\]
\begin{enumerate}
\item[(a)]  For which $n$ is a row of $A$ a vector in $\R^n$? \begin{prompt}\[n = \answer{4}\]\end{prompt}.
\item[(b)]  What is the $2^{nd}$ column of $A$?
  \begin{prompt}
    \[
      \left(\begin{array}{r} \answer{-1} \\ \answer{4} \\ \answer{-3} \end{array} \right)
    \]
  \end{prompt}
\item[(c)] Let $a_{ij}$ be the entry of $A$ in the $i^{th}$ row
  and the $j^{th}$ column.  What is $a_{23}-a_{31}$?
  \begin{prompt}
    \[
      a_{23}-a_{31} = \answer{-13}.
    \]
  \end{prompt}
\end{enumerate}

\begin{solution}
\ans
(a) The number of entries in a row is the number of columns.  Thus, $n = 4$; \\
(b) $\left(\begin{array}{r} -1 \\ 4 \\ -3 \end{array} \right)$;
(c) $a_{23}-a_{31} =  -7 - 6 = -13$.

\end{solution}
\end{exercise}





\end{document}
