\documentclass{ximera}

\usepackage{pgf,tikz}
\usepackage{mathrsfs}
\usetikzlibrary{shapes,arrows}
\usepackage{framed}
\pgfplotsset{compat=1.13}

\graphicspath{
  {./}
  {ximeraTutorial/}
}

\newenvironment{sectionOutcomes}{}{}


% Below is the preamble from the textbook in /laode, by Golubitsky and Dellnitz
% ... with problematic stuff commented out
% Followed by the hw-preamble from worksheet-builder

%\usepackage{ulem}
\usepackage[normalem]{ulem}

\epstopdfsetup{outdir=./}

\usepackage{morewrites}
\makeatletter
\newcommand\subfile[1]{%
\renewcommand{\input}[1]{}%
\begingroup\skip@preamble\otherinput{#1}\endgroup\par\vspace{\topsep}
\let\input\otherinput}
\makeatother

\newcommand{\EXER}{}
\newcommand{\includeexercises}{\EXER\directlua{dofile(kpse.find_file("exercises","lua"))}}

\newenvironment{computerExercise}{\begin{exercise}}{\end{exercise}}

%\newcounter{ccounter}
%\setcounter{ccounter}{1}
%\newcommand{\Chapter}[1]{\setcounter{chapter}{\arabic{ccounter}}\chapter{#1}\addtocounter{ccounter}{1}}

%\newcommand{\section}[1]{\section{#1}\setcounter{thm}{0}\setcounter{equation}{0}}

%\renewcommand{\theequation}{\arabic{chapter}.\arabic{section}.\arabic{equation}}
%\renewcommand{\thefigure}{\arabic{chapter}.\arabic{figure}}
%\renewcommand{\thetable}{\arabic{chapter}.\arabic{table}}

%\newcommand{\Sec}[2]{\section{#1}\markright{\arabic{ccounter}.\arabic{section}.#2}\setcounter{equation}{0}\setcounter{thm}{0}\setcounter{figure}{0}}
  
\newcommand{\Sec}[2]{\section{#1}}

\setcounter{secnumdepth}{2}
%\setcounter{secnumdepth}{1} 

%\newcounter{THM}
%\renewcommand{\theTHM}{\arabic{chapter}.\arabic{section}}

\newcommand{\trademark}{{R\!\!\!\!\!\bigcirc}}
%\newtheorem{exercise}{}

\newcommand{\dfield}{{\sf SlopeField}}

\newcommand{\pplane}{{\sf PhasePlane}}

\newcommand{\PPLANE}{{\sf PHASEPLANE}}

% BADBAD: \newcommand{\Bbb}{\bf}. % Package amsfonts Warning: Obsolete command \Bbb; \mathbb should be used instead.

\newcommand{\R}{\mbox{$\mathbb{R}$}}
\let\C\relax
\newcommand{\C}{\mbox{$\mathbb{C}$}}
\newcommand{\Z}{\mbox{$\mathbb{Z}$}}
\newcommand{\N}{\mbox{$\mathbb{N}$}}
\newcommand{\D}{\mbox{{\bf D}}}

\newcommand{\WW}{\mathcal{W}}

\usepackage{amssymb}
%\newcommand{\qed}{\hfill\mbox{\raggedright$\square$} \vspace{1ex}}
%\newcommand{\proof}{\noindent {\bf Proof:} \hspace{0.1in}}

\newcommand{\setmin}{\;\mbox{--}\;}
\newcommand{\Matlab}{{M\small{AT\-LAB}} }
\newcommand{\Matlabp}{{M\small{AT\-LAB}}}
\newcommand{\computer}{\Matlab Instructions}
\renewcommand{\computer}{M\small{ATLAB} Instructions}
\newcommand{\half}{\mbox{$\frac{1}{2}$}}
\newcommand{\compose}{\raisebox{.15ex}{\mbox{{\scriptsize$\circ$}}}}
\newcommand{\AND}{\quad\mbox{and}\quad}
\newcommand{\vect}[2]{\left(\begin{array}{c} #1_1 \\ \vdots \\
 #1_{#2}\end{array}\right)}
\newcommand{\mattwo}[4]{\left(\begin{array}{rr} #1 & #2\\ #3
&#4\end{array}\right)}
\newcommand{\mattwoc}[4]{\left(\begin{array}{cc} #1 & #2\\ #3
&#4\end{array}\right)}
\newcommand{\vectwo}[2]{\left(\begin{array}{r} #1 \\ #2\end{array}\right)}
\newcommand{\vectwoc}[2]{\left(\begin{array}{c} #1 \\ #2\end{array}\right)}

\newcommand{\ignore}[1]{}


\newcommand{\inv}{^{-1}}
\newcommand{\CC}{{\cal C}}
\newcommand{\CCone}{\CC^1}
\newcommand{\Span}{{\rm span}}
\newcommand{\rank}{{\rm rank}}
\newcommand{\trace}{{\rm tr}}
\newcommand{\RE}{{\rm Re}}
\newcommand{\IM}{{\rm Im}}
\newcommand{\nulls}{{\rm null\;space}}

\newcommand{\dps}{\displaystyle}
\newcommand{\arraystart}{\renewcommand{\arraystretch}{1.8}}
\newcommand{\arrayfinish}{\renewcommand{\arraystretch}{1.2}}
\newcommand{\Start}[1]{\vspace{0.08in}\noindent {\bf Section~\ref{#1}}}
\newcommand{\exer}[1]{\noindent {\bf \ref{#1}}}
\newcommand{\ans}{\textbf{Answer:} }
\newcommand{\matthree}[9]{\left(\begin{array}{rrr} #1 & #2 & #3 \\ #4 & #5 & #6
\\ #7 & #8 & #9\end{array}\right)}
\newcommand{\cvectwo}[2]{\left(\begin{array}{c} #1 \\ #2\end{array}\right)}
\newcommand{\cmatthree}[9]{\left(\begin{array}{ccc} #1 & #2 & #3 \\ #4 & #5 &
#6 \\ #7 & #8 & #9\end{array}\right)}
\newcommand{\vecthree}[3]{\left(\begin{array}{r} #1 \\ #2 \\
#3\end{array}\right)}
\newcommand{\cvecthree}[3]{\left(\begin{array}{c} #1 \\ #2 \\
#3\end{array}\right)}
\newcommand{\cmattwo}[4]{\left(\begin{array}{cc} #1 & #2\\ #3
&#4\end{array}\right)}

\newcommand{\Matrix}[1]{\ensuremath{\left(\begin{array}{rrrrrrrrrrrrrrrrrr} #1 \end{array}\right)}}

\newcommand{\Matrixc}[1]{\ensuremath{\left(\begin{array}{cccccccccccc} #1 \end{array}\right)}}



\renewcommand{\labelenumi}{\theenumi}
\newenvironment{enumeratea}%
{\begingroup
 \renewcommand{\theenumi}{\alph{enumi}}
 \renewcommand{\labelenumi}{(\theenumi)}
 \begin{enumerate}}
 {\end{enumerate}
 \endgroup}

\newcounter{help}
\renewcommand{\thehelp}{\thesection.\arabic{equation}}

%\newenvironment{equation*}%
%{\renewcommand\endequation{\eqno (\theequation)* $$}%
%   \begin{equation}}%
%   {\end{equation}\renewcommand\endequation{\eqno \@eqnnum
%$$\global\@ignoretrue}}

\author{Martin Golubitsky and Michael Dellnitz}

%\newenvironment{matlabEquation}%
%{\renewcommand\endequation{\eqno (\theequation*) $$}%
%   \begin{equation}}%
%   {\end{equation}\renewcommand\endequation{\eqno \@eqnnum
% $$\global\@ignoretrue}}

\newcommand{\soln}{\textbf{Solution:} }
\newcommand{\exercap}[1]{\centerline{Figure~\ref{#1}}}
\newcommand{\exercaptwo}[1]{\centerline{Figure~\ref{#1}a\hspace{2.1in}
Figure~\ref{#1}b}}
\newcommand{\exercapthree}[1]{\centerline{Figure~\ref{#1}a\hspace{1.2in}
Figure~\ref{#1}b\hspace{1.2in}Figure~\ref{#1}c}}
\newcommand{\para}{\hspace{0.4in}}

%\usepackage{ifluatex}
%\ifluatex
%\ifcsname displaysolutions\endcsname%
%\else
%\renewenvironment{solution}{\suppress}{\endsuppress}
%\fi
%\else
%\renewenvironment{solution}{}{}
%\fi
%
%\ifcsname answer\endcsname
%\renewcommand{\answer}{}
%\fi

%\ifxake
%\newenvironment{matlabEquation}{\begin{equation}}{\end{equation}}
%\else
\newenvironment{matlabEquation}%
{\let\oldtheequation\theequation\renewcommand{\theequation}{\oldtheequation*}\begin{equation}}%
  {\end{equation}\let\theequation\oldtheequation}
%\fi

%\makeatother

\newcommand{\RED}[1]{{\color{red}{#1}}} 


%%
%%
%% Worksheet-builder preamble
%%
%%

\usepackage{xcolor}
\renewenvironment{solution}{\color{blue}}{\color{black}}
\renewenvironment{computerExercise}{\begin{exercise}\textsc{(matlab)} }{\end{exercise}}

%\usepackage{environ}
%\RenewEnviron{prompt}{}
%\RenewEnviron{hint}{}
%\RenewEnviron{multipleChoice}{}
%\RenewEnviron{feedback}{}

\renewcommand{\ans}{\noindent\textbf{Answer: }}
\renewcommand{\soln}{\noindent\textbf{Solution: }}

%\renewcommand{\answer}[2][]{#2}

% if you want to hide solutions, uncomment the following
%\usepackage{comment}\excludecomment{solution}

\def\isitmatlab{}
\newcommand{\matlab}{\def\isitmatlab{ (MATLAB)}}

\makeatletter
\newcommand{\exerciselabel}[2]{\textbf{\textsection #2, Exercise #1\isitmatlab.}\def\@currentlabel{#1}\def\isitmatlab{}}
\makeatother

\newcounter{problemx}
\newcommand{\problemlabel}{\refstepcounter{problemx}\section*{Problem \arabic{problemx}}}
\newcommand{\matlabproblemlabel}{\refstepcounter{problemx}\section*{Problem \arabic{problemx} (MATLAB)}}
%\newcommand{\problemlabel}{\refstepcounter{problem}\section*{Problem}}
%\newcommand{\matlabproblemlabel}{\refstepcounter{problem}\section*{Problem (MATLAB)}}




\title{Constructing Subspaces}
\author{\phantom{Dr. Golubitsky}}
\date{Due: TBA}


%\makeatletter
%\newlabel{c5.2.1a}{{1}{133}}
%\newlabel{c5.2.1d}{{4}{133}}
%\newlabel{c5.2.2a}{{5}{133}}
%\newlabel{c5.2.2d}{{8}{133}}
%\newlabel{c5.2.6a}{{16}{134}}
%\newlabel{c5.2.6d}{{19}{134}}
%\makeatother
\begin{document}
\begin{abstract}
Constructing subspaces.
\end{abstract}
\maketitle



\problemlabel

\noindent In Exercises~\ref{c5.2.1a} -- \ref{c5.2.1d} a single equation in
three variables is given.  For each equation write the subspace of solutions
in $\R^3$ as the span of two vectors in $\R^3$.


\exerciselabel{2}{5.2}\begin{exercise} \label{c5.2.1b}
$x - y + 3z = 0$.

\begin{solution}
\ans The subspace of solutions can be spanned by the vectors 
$(1,1,0)^t$ and $(-3,0,1)^t$.

\soln All solutions to $x - y + 3z = 0$ can be written in the form
\[
\vecthree{x}{y}{z} = \cvecthree{y-3z}{y}{z}
= y\vecthree{1}{1}{0} + z\vecthree{-3}{0}{1}.
\]

\end{solution}
\end{exercise}


%%%%%%%%%%%%%%%%%%%%%%%%%%%%%%%%%%%%%%%%%%%%%%%%%%%%%%%%%%%%%%%%


\problemlabel

\noindent In Exercises~\ref{c5.2.2a} -- \ref{c5.2.2d} each of the
given matrices is in reduced echelon form.  Write solutions of the
corresponding homogeneous system of linear equations as a span of vectors.


\exerciselabel{7}{5.2}\begin{exercise} \label{c5.2.2c}
$A = \left(\begin{array}{rrr} 1 & 0 & 2 \\
        0 & 1 & 1\end{array}\right)$.

\begin{solution}

\ans The subspace of solutions to $Ax = 0$ is spanned by the vector
$(-2,-1,1)^t$.

\soln Let $x = (x_1,x_2,x_3)$ be a solution to $Ax = 0$.  All solutions
to this equation have the form
\[
\vecthree{x_1}{x_2}{x_3} = \vecthree{-2x_3}{-x_3}{x_3} =
x_3\vecthree{-2}{-1}{1}.
\]

\end{solution}
\end{exercise}


%%%%%%%%%%%%%%%%%%%%%%%%%%%%%%%%%%%%%%%%%%%%%%%%%%%%%%%%%%%%%%%%


\problemlabel

\exerciselabel{8}{5.2}\begin{exercise} \label{c5.2.2d}
$B = \left(\begin{array}{rrrrrr} 1 & -1 & 0 & 5 & 0 & 0\\
        0 & 0 & 1 & 2 & 0 & 2\\
        0 & 0 & 0 & 0 & 1 & 2\end{array}\right)$.

\begin{solution}

\ans The subspace of solutions to $Bx = 0$ is spanned by the vectors
\[
\left(\begin{array}{r} 1 \\ 1 \\ 0 \\ 0 \\ 0 \\ 0 \end{array}\right), \quad
\left(\begin{array}{r} -5 \\ 0 \\ -2 \\ 1 \\ 0 \\ 0 \end{array}\right), \quad
\left(\begin{array}{r} 0 \\ 0 \\ -2 \\ 0 \\ -2 \\ 1 \end{array}\right).
\]

\soln Let $x = (x_1,\dots,x_6)$ be a solution to $Bx = 0$.  All solutions
to this equation have the form
\[
\left(\begin{array}{r} x_1 \\ x_2 \\ x_3 \\ x_4 \\ x_5 \\ x_6
\end{array}\right) =
\left(\begin{array}{c} x_2 - 5x_4 \\ x_2 \\ -2x_4 - 2x_6 \\ x_4 \\ -2x_6
\\ x_6 \end{array}\right) =
x_2\left(\begin{array}{r} 1 \\ 1 \\ 0 \\ 0 \\ 0 \\ 0 \end{array}\right) +
x_4\left(\begin{array}{r} -5 \\ 0 \\ -2 \\ 1 \\ 0 \\ 0 \end{array}\right) +
x_6\left(\begin{array}{r} 0 \\ 0 \\ -2 \\ 0 \\ -2 \\ 1 \end{array}\right).
\]


\end{solution}
\end{exercise}


%%%%%%%%%%%%%%%%%%%%%%%%%%%%%%%%%%%%%%%%%%%%%%%%%%%%%%%%%%%%%%%%


\problemlabel



\exerciselabel{9}{5.2}\begin{exercise} \label{c5.2.3}
Write a system of two linear equations of the form $Ax=0$ where
$A$ is a $2\times 4$ matrix whose subspace of solutions in $\R^4$
is the span of the two vectors
\[
v_1 = \left(\begin{array}{r} 1 \\ -1 \\ 0 \\  0 \end{array}\right) \AND
v_2 = \left(\begin{array}{r} 0 \\  0 \\ 1 \\ -1 \end{array}\right).
\]

\begin{solution}

\ans The matrix $A$ whose subspace of solutions in $\R^4$ is the span of
$v_1$ and $v_2$ is
\[
A = \left(\begin{array}{rrrr} 1 & 1 & 0 & 0 \\ 0 & 0 & 1 & 1
\end{array}\right).
\]

\soln Note that all vectors $x$ in the spanning set of $v_1$ and $v_2$
are of the form:
\[
x = \left(\begin{array}{r} x_1 \\ x_2 \\ x_3 \\ x_4
\end{array}\right)
= a\left(\begin{array}{r} 1 \\ -1 \\ 0 \\ 0 \end{array}\right) + 
b\left(\begin{array}{r} 0 \\ 0 \\ 1 \\ -1 \end{array}\right) =
\left(\begin{array}{r} a \\ -a \\ b \\ -b \end{array}\right).
\]
Therefore, $x_1 = -x_2$ and $x_3 = -x_4$.  So,
\[
\begin{array}{rrrrrrrrl}
x_1 & + & x_2 & & & & & = & 0 \\
& & & & x_3 & + & x_4 & = & 0. \end{array}
\]
The matrix of this system is $A$.

\end{solution}
\end{exercise}


%%%%%%%%%%%%%%%%%%%%%%%%%%%%%%%%%%%%%%%%%%%%%%%%%%%%%%%%%%%%%%%%


\problemlabel

\exerciselabel{15}{5.2}\begin{exercise} \label{c5.2.5}
Is $(2,20,0)$ in the span of $w_1=(1,1,3)$ and $w_2=(1,4,2)$?
Answer this question by setting up a system of linear equations
and solving that system by row reducing the associated augmented
matrix\index{matrix!augmented}.

\begin{solution}

\ans The vector $(2,20,0)^t$ is in the span of $w_1$ and $w_2$. 
Specifically, $v = -4w_1 + 6w_2$.

\soln Note that, for some real numbers $a$ and $b$,
\[
(2,20,0)^t = aw_1 + bw_2 = a(1,1,3)^t + b(1,4,2)^t
\]
if $v$ is in the span of $w_1$ and $w_2$.
This corresponds to the linear system
\[
\begin{array}{rrrrr}
a & + & b & = 2 \\
a & + & 4b & = 20 \\
3a & + & 2b & = 0 \end{array}
\]
To find $a$ and $b$, row reduce the augmented matrix of the system:
\[
\left(\begin{array}{rr|r} 1 & 1 & 2 \\ 1 & 4 & 20 \\
3 & 2 & 0 \end{array}\right) \longrightarrow
\left(\begin{array}{rr|r} 1 & 0 & -4 \\ 0 & 1 & 6 \\
0 & 0 & 0 \end{array}\right).
\]
The system is consistent; $a = -4$ and $b = 6$.

\end{solution}
\end{exercise}


%%%%%%%%%%%%%%%%%%%%%%%%%%%%%%%%%%%%%%%%%%%%%%%%%%%%%%%%%%%%%%%%


\problemlabel

\noindent In Exercises~\ref{c5.2.6a} -- \ref{c5.2.6d} let $W\subset\CCone$
be the subspace spanned by the two polynomials $x_1(t) = 1$ and
$x_2(t)=t^2$.  For the given function $y(t)$ decide whether or not $y(t)$
is an element of $W$.  Furthermore, if $y(t)\in W$, determine whether the set
$\{y(t),x_2(t)\}$ is a spanning set for $W$.


\exerciselabel{16}{5.2}\begin{exercise} \label{c5.2.6a}
$y(t) = 1-t^2$,

\begin{solution}

\ans The function $y(t) = 1 - t^2$ is an element of $W$ and the set
$\{y(t),x_2(t)\}$ is a spanning set for $W$.



\soln The space $W$ equals $\Span\{x_1(t),x_2(t)\}$ where $x_1(t)=1$ and 
$x_2(t)=t^2$.  To show that $y(t)$ is an element of $W$, let
$a = 1$ and $b = -1$, and compute
\[
ax_1(t) + bx_2(t) = x_1(t) - x_2(t) = 1 - t^2 = y(t). 
\]
To show that $\{y(t),x_2(t)\}$ is a spanning set for $W$, rewrite every
linear combination of $x_1(t)$ and $x_2(t)$ in terms of $y(t)$ and $x_2(t)$, 
as follows:
\[ 
ax_1(t) + bx_2(t) = a + bt^2 = a(1 - t^2) + (a + b)t^2
= ay(t) + (a + b)x_2(t). 
\]

\end{solution}
\end{exercise}


%%%%%%%%%%%%%%%%%%%%%%%%%%%%%%%%%%%%%%%%%%%%%%%%%%%%%%%%%%%%%%%%


\problemlabel

\exerciselabel{25}{5.2}\begin{exercise} \label{c5.2.10}
Let $Ax=b$ be a system of $m$ linear equations in $n$ unknowns,
and let $r=\rank(A)$ and $s=\rank(A|b)$.  Suppose that this system
has a unique solution.  What can you say about the relative
magnitudes of $m,n,r,s$?

\begin{solution}

\ans The relationship of the constants is $m \geq n = r = s$.

\soln The rank of matrix $A$ cannot be greater than the rank of matrix
$(A|b)$, since $(A|b)$ consists of $A$ plus one column.  The rank of $A$
is the number of pivots in the row reduced matrix.  $(A|b)$ can be row 
reduced through the same operations, and will have either the same number
of pivots as $A$ or, if there is a pivot in the last column, one more
pivot than $A$.  Since the system has a unique solution, it is consistent,
and therefore $(A|b)$ cannot have a pivot in the $(n + 1)^{st}$ column, so
$r = \rank(A) = \rank(A|b) = s$.

\para The set of solutions is parameterized by $n - r$ parameters,
where $n$ is the number of columns of $A$.  Since there is a unique
solution, the set of solutions is parameterized by $0$ parameters,
so $n = r$.

\para The number $m$ of rows of the matrix must be greater than or
equal to $n$ in order for the system to have a unique solution, since
there must be $n$ pivots, and each pivot must be in a separate row.



\end{solution}
\end{exercise}


%%%%%%%%%%%%%%%%%%%%%%%%%%%%%%%%%%%%%%%%%%%%%%%%%%%%%%%%%%%%%%%%



\end{document}
