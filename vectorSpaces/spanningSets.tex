\documentclass{ximera}

\input{../preamble}


\title{Spanning Sets}
\author{\phantom{Dr. Golubitsky}}
\date{Due: TBA}

%\makeatletter
%\newlabel{c5.3.1a}{{1}{137}}
%\newlabel{c5.3.1c}{{3}{137}}
%\newlabel{c5.3.4a}{{6}{137}}
%\newlabel{c5.3.4c}{{8}{137}}
%\makeatother
\begin{document}
\begin{abstract}
Spanning Sets
\end{abstract}
\maketitle




\matlabproblemlabel

\noindent In Exercises~\ref{c5.3.1a} -- \ref{c5.3.1c} use the {\tt null}
command in \Matlab to find all the solutions of the linear system of
equations $Ax=0$.


\exerciselabel{1}{5.3}\begin{computerExercise} \label{c5.3.1a}
\begin{matlabEquation} \label{e:BCDa}
          A=    \left(\begin{array}{cccc}
                -4 & 0 & -4 & 3\\
                -4 & 1 & -1 & 1
                \end{array}\right) \quad
\end{matlabEquation}

\begin{solution}

Type {\tt null(A)} in \Matlab to find that the set of solutions to
$Ax = 0$ is spanned by the vectors
\[
\left(\begin{array}{r} 0.3225 \\ 0.8931 \\ -0.0992 \\ 0.2977
\end{array}\right) \AND \left(\begin{array}{r} 0 \\ -0.1961 \\
0.5883 \\ 0.7845 \end{array}\right).
\]

\end{solution}
\end{computerExercise}


%%%%%%%%%%%%%%%%%%%%%%%%%%%%%%%%%%%%%%%%%%%%%%%%%%%%%%%%%%%%%%%%


\matlabproblemlabel

\exerciselabel{3}{5.3}\begin{computerExercise} \label{c5.3.1c}
\begin{matlabEquation} \label{e:BCDc}
          A=      \left(\begin{array}{rrr}
               1  &  1  &  2\\
              -1  &  2  & -1
                \end{array}\right).
\end{matlabEquation}

\begin{solution}
The set of solutions to $Ax = 0$ is spanned by the vector
\[
\vecthree{-0.8452}{-0.1690}{0.5071}.
\]


\end{solution}
\end{computerExercise}


%%%%%%%%%%%%%%%%%%%%%%%%%%%%%%%%%%%%%%%%%%%%%%%%%%%%%%%%%%%%%%%%


\matlabproblemlabel

\noindent In Exercises~\ref{c5.3.4a} -- \ref{c5.3.4c} let $W\subset\R^5$
be the subspace spanned by the vectors
\begin{matlabEquation}\label{MATLAB:65}
     w_1=(2,0,-1,3,4),\quad w_2=(1,0,0,-1,2),\quad w_3=(0,1,0,0,-1).
\end{matlabEquation}
Use \Matlab to decide whether the given vectors are elements of $W$.


\exerciselabel{6}{5.3}\begin{computerExercise} \label{c5.3.4a}
$v_1=(2,1,-2,8,3)$.

\begin{solution}
\ans Vector $v_1$ is an element of $W$.

\soln The vector $v_1$ is an element of $W$ if there exist scalars $a$,
$b$, and $c$ such that
\[
aw_1 + bw_2 + cw_3 = v_1.
\]
Using \Matlab, create the matrix {\tt A = [w1' w2' w3']}, which has
$w_1$, $w_2$, and $w_3$ as its columns.  Then create the augmented
matrix {\tt aug1 = [A v1']}.  The command {\tt rref(aug1)} yields
\begin{verbatim}
ans =
     1     0     0     2
     0     1     0    -2
     0     0     1     1
     0     0     0     0
     0     0     0     0
\end{verbatim}
Since there is no pivot point in the last column, the linear system
$aw_1 + bw_2 + cw_3 = v_1$ is consistent, and $v_1 = 2w_1 - 2w_2 + w_3$.

\end{solution}
\end{computerExercise}


%%%%%%%%%%%%%%%%%%%%%%%%%%%%%%%%%%%%%%%%%%%%%%%%%%%%%%%%%%%%%%%%




\end{document}
