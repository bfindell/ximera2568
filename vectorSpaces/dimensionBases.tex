\documentclass{ximera}

\usepackage{pgf,tikz}
\usepackage{mathrsfs}
\usetikzlibrary{shapes,arrows}
\usepackage{framed}
\pgfplotsset{compat=1.13}

\graphicspath{
  {./}
  {ximeraTutorial/}
}

\newenvironment{sectionOutcomes}{}{}


% Below is the preamble from the textbook in /laode, by Golubitsky and Dellnitz
% ... with problematic stuff commented out
% Followed by the hw-preamble from worksheet-builder

%\usepackage{ulem}
\usepackage[normalem]{ulem}

\epstopdfsetup{outdir=./}

\usepackage{morewrites}
\makeatletter
\newcommand\subfile[1]{%
\renewcommand{\input}[1]{}%
\begingroup\skip@preamble\otherinput{#1}\endgroup\par\vspace{\topsep}
\let\input\otherinput}
\makeatother

\newcommand{\EXER}{}
\newcommand{\includeexercises}{\EXER\directlua{dofile(kpse.find_file("exercises","lua"))}}

\newenvironment{computerExercise}{\begin{exercise}}{\end{exercise}}

%\newcounter{ccounter}
%\setcounter{ccounter}{1}
%\newcommand{\Chapter}[1]{\setcounter{chapter}{\arabic{ccounter}}\chapter{#1}\addtocounter{ccounter}{1}}

%\newcommand{\section}[1]{\section{#1}\setcounter{thm}{0}\setcounter{equation}{0}}

%\renewcommand{\theequation}{\arabic{chapter}.\arabic{section}.\arabic{equation}}
%\renewcommand{\thefigure}{\arabic{chapter}.\arabic{figure}}
%\renewcommand{\thetable}{\arabic{chapter}.\arabic{table}}

%\newcommand{\Sec}[2]{\section{#1}\markright{\arabic{ccounter}.\arabic{section}.#2}\setcounter{equation}{0}\setcounter{thm}{0}\setcounter{figure}{0}}
  
\newcommand{\Sec}[2]{\section{#1}}

\setcounter{secnumdepth}{2}
%\setcounter{secnumdepth}{1} 

%\newcounter{THM}
%\renewcommand{\theTHM}{\arabic{chapter}.\arabic{section}}

\newcommand{\trademark}{{R\!\!\!\!\!\bigcirc}}
%\newtheorem{exercise}{}

\newcommand{\dfield}{{\sf SlopeField}}

\newcommand{\pplane}{{\sf PhasePlane}}

\newcommand{\PPLANE}{{\sf PHASEPLANE}}

% BADBAD: \newcommand{\Bbb}{\bf}. % Package amsfonts Warning: Obsolete command \Bbb; \mathbb should be used instead.

\newcommand{\R}{\mbox{$\mathbb{R}$}}
\let\C\relax
\newcommand{\C}{\mbox{$\mathbb{C}$}}
\newcommand{\Z}{\mbox{$\mathbb{Z}$}}
\newcommand{\N}{\mbox{$\mathbb{N}$}}
\newcommand{\D}{\mbox{{\bf D}}}

\newcommand{\WW}{\mathcal{W}}

\usepackage{amssymb}
%\newcommand{\qed}{\hfill\mbox{\raggedright$\square$} \vspace{1ex}}
%\newcommand{\proof}{\noindent {\bf Proof:} \hspace{0.1in}}

\newcommand{\setmin}{\;\mbox{--}\;}
\newcommand{\Matlab}{{M\small{AT\-LAB}} }
\newcommand{\Matlabp}{{M\small{AT\-LAB}}}
\newcommand{\computer}{\Matlab Instructions}
\renewcommand{\computer}{M\small{ATLAB} Instructions}
\newcommand{\half}{\mbox{$\frac{1}{2}$}}
\newcommand{\compose}{\raisebox{.15ex}{\mbox{{\scriptsize$\circ$}}}}
\newcommand{\AND}{\quad\mbox{and}\quad}
\newcommand{\vect}[2]{\left(\begin{array}{c} #1_1 \\ \vdots \\
 #1_{#2}\end{array}\right)}
\newcommand{\mattwo}[4]{\left(\begin{array}{rr} #1 & #2\\ #3
&#4\end{array}\right)}
\newcommand{\mattwoc}[4]{\left(\begin{array}{cc} #1 & #2\\ #3
&#4\end{array}\right)}
\newcommand{\vectwo}[2]{\left(\begin{array}{r} #1 \\ #2\end{array}\right)}
\newcommand{\vectwoc}[2]{\left(\begin{array}{c} #1 \\ #2\end{array}\right)}

\newcommand{\ignore}[1]{}


\newcommand{\inv}{^{-1}}
\newcommand{\CC}{{\cal C}}
\newcommand{\CCone}{\CC^1}
\newcommand{\Span}{{\rm span}}
\newcommand{\rank}{{\rm rank}}
\newcommand{\trace}{{\rm tr}}
\newcommand{\RE}{{\rm Re}}
\newcommand{\IM}{{\rm Im}}
\newcommand{\nulls}{{\rm null\;space}}

\newcommand{\dps}{\displaystyle}
\newcommand{\arraystart}{\renewcommand{\arraystretch}{1.8}}
\newcommand{\arrayfinish}{\renewcommand{\arraystretch}{1.2}}
\newcommand{\Start}[1]{\vspace{0.08in}\noindent {\bf Section~\ref{#1}}}
\newcommand{\exer}[1]{\noindent {\bf \ref{#1}}}
\newcommand{\ans}{\textbf{Answer:} }
\newcommand{\matthree}[9]{\left(\begin{array}{rrr} #1 & #2 & #3 \\ #4 & #5 & #6
\\ #7 & #8 & #9\end{array}\right)}
\newcommand{\cvectwo}[2]{\left(\begin{array}{c} #1 \\ #2\end{array}\right)}
\newcommand{\cmatthree}[9]{\left(\begin{array}{ccc} #1 & #2 & #3 \\ #4 & #5 &
#6 \\ #7 & #8 & #9\end{array}\right)}
\newcommand{\vecthree}[3]{\left(\begin{array}{r} #1 \\ #2 \\
#3\end{array}\right)}
\newcommand{\cvecthree}[3]{\left(\begin{array}{c} #1 \\ #2 \\
#3\end{array}\right)}
\newcommand{\cmattwo}[4]{\left(\begin{array}{cc} #1 & #2\\ #3
&#4\end{array}\right)}

\newcommand{\Matrix}[1]{\ensuremath{\left(\begin{array}{rrrrrrrrrrrrrrrrrr} #1 \end{array}\right)}}

\newcommand{\Matrixc}[1]{\ensuremath{\left(\begin{array}{cccccccccccc} #1 \end{array}\right)}}



\renewcommand{\labelenumi}{\theenumi}
\newenvironment{enumeratea}%
{\begingroup
 \renewcommand{\theenumi}{\alph{enumi}}
 \renewcommand{\labelenumi}{(\theenumi)}
 \begin{enumerate}}
 {\end{enumerate}
 \endgroup}

\newcounter{help}
\renewcommand{\thehelp}{\thesection.\arabic{equation}}

%\newenvironment{equation*}%
%{\renewcommand\endequation{\eqno (\theequation)* $$}%
%   \begin{equation}}%
%   {\end{equation}\renewcommand\endequation{\eqno \@eqnnum
%$$\global\@ignoretrue}}

\author{Martin Golubitsky and Michael Dellnitz}

%\newenvironment{matlabEquation}%
%{\renewcommand\endequation{\eqno (\theequation*) $$}%
%   \begin{equation}}%
%   {\end{equation}\renewcommand\endequation{\eqno \@eqnnum
% $$\global\@ignoretrue}}

\newcommand{\soln}{\textbf{Solution:} }
\newcommand{\exercap}[1]{\centerline{Figure~\ref{#1}}}
\newcommand{\exercaptwo}[1]{\centerline{Figure~\ref{#1}a\hspace{2.1in}
Figure~\ref{#1}b}}
\newcommand{\exercapthree}[1]{\centerline{Figure~\ref{#1}a\hspace{1.2in}
Figure~\ref{#1}b\hspace{1.2in}Figure~\ref{#1}c}}
\newcommand{\para}{\hspace{0.4in}}

%\usepackage{ifluatex}
%\ifluatex
%\ifcsname displaysolutions\endcsname%
%\else
%\renewenvironment{solution}{\suppress}{\endsuppress}
%\fi
%\else
%\renewenvironment{solution}{}{}
%\fi
%
%\ifcsname answer\endcsname
%\renewcommand{\answer}{}
%\fi

%\ifxake
%\newenvironment{matlabEquation}{\begin{equation}}{\end{equation}}
%\else
\newenvironment{matlabEquation}%
{\let\oldtheequation\theequation\renewcommand{\theequation}{\oldtheequation*}\begin{equation}}%
  {\end{equation}\let\theequation\oldtheequation}
%\fi

%\makeatother

\newcommand{\RED}[1]{{\color{red}{#1}}} 


%%
%%
%% Worksheet-builder preamble
%%
%%

\usepackage{xcolor}
\renewenvironment{solution}{\color{blue}}{\color{black}}
\renewenvironment{computerExercise}{\begin{exercise}\textsc{(matlab)} }{\end{exercise}}

%\usepackage{environ}
%\RenewEnviron{prompt}{}
%\RenewEnviron{hint}{}
%\RenewEnviron{multipleChoice}{}
%\RenewEnviron{feedback}{}

\renewcommand{\ans}{\noindent\textbf{Answer: }}
\renewcommand{\soln}{\noindent\textbf{Solution: }}

%\renewcommand{\answer}[2][]{#2}

% if you want to hide solutions, uncomment the following
%\usepackage{comment}\excludecomment{solution}

\def\isitmatlab{}
\newcommand{\matlab}{\def\isitmatlab{ (MATLAB)}}

\makeatletter
\newcommand{\exerciselabel}[2]{\textbf{\textsection #2, Exercise #1\isitmatlab.}\def\@currentlabel{#1}\def\isitmatlab{}}
\makeatother

\newcounter{problemx}
\newcommand{\problemlabel}{\refstepcounter{problemx}\section*{Problem \arabic{problemx}}}
\newcommand{\matlabproblemlabel}{\refstepcounter{problemx}\section*{Problem \arabic{problemx} (MATLAB)}}
%\newcommand{\problemlabel}{\refstepcounter{problem}\section*{Problem}}
%\newcommand{\matlabproblemlabel}{\refstepcounter{problem}\section*{Problem (MATLAB)}}




\title{Dimension and Bases}
\author{\phantom{Dr. Golubitsky}}
\date{Due: TBA}




%\makeatletter
%\newlabel{basis=span+indep}{{5.5.3}{142}}
%\newlabel{L:computerank}{{5.5.4}{142}}
%\newlabel{extendindep}{{5.6.4}{148}}
%\makeatother

\begin{document}
\begin{abstract}
Dimension and Bases
\end{abstract}
\maketitle



\problemlabel



\exerciselabel{1}{5.5}\begin{exercise} \label{c5.5.1}
Show that ${\cal U}=\{u_1,u_2,u_3\}$ where
\[
u_1=(1,1,0) \quad u_2=(0,1,0) \quad u_3=(-1,0,1)
\]
is a basis for $\R^3$.

\begin{solution}

By Theorem~\ref{basis=span+indep},
${\cal U}$ is a basis for $\R^3$ if the vectors of ${\cal U}$ are
linearly independent and span $\R^3$.  By Lemma~\ref{L:computerank},
the dimension of ${\cal U}$ is equal to the rank of the matrix whose
rows are $u_1$, $u_2$, and $u_3$.  Row reduce this matrix:
\[
\matthree{1}{1}{0}{0}{1}{0}{-1}{0}{1} \longrightarrow
\matthree{1}{0}{0}{0}{1}{0}{0}{0}{1}.
\]
So $\dim({\cal U}) = 3 = \dim(\R^3)$, and we need now only show that
$u_1$, $u_2$, and $u_3$ are linearly independent, which we can do by
row reducing the matrix whose columns are the vectors of ${\cal U}$ as
follows:
\[
\matthree{1}{0}{-1}{1}{1}{0}{0}{0}{1} \longrightarrow
\matthree{1}{0}{0}{0}{1}{0}{0}{0}{1}.
\]
Therefore, there is no nonzero solution to the equation
${\cal U}r = 0$, so the vectors of ${\cal U}$ are linearly independent
and ${\cal U}$ is a basis for $\R^3$.

\end{solution}
\end{exercise}


%%%%%%%%%%%%%%%%%%%%%%%%%%%%%%%%%%%%%%%%%%%%%%%%%%%%%%%%%%%%%%%%


\problemlabel

\exerciselabel{3}{5.5}\begin{exercise} \label{c5.5.2}
Let $S=\Span\{v_1,v_2,v_3\}$ where
\[
v_1=(1,0,-1,0) \quad v_2=(0,1,1,1) \quad v_3=(5,4,-1,4).
\]
Find the dimension of $S$ and find a basis for $S$.

\begin{solution}

\ans The dimension of $S$ is 2, and vectors $v_1$ and $v_2$ form a
basis for $S$.

\soln Row reduce the matrix $A$ whose rows are $v_1$, $v_2$, and $v_3$. 
By Lemma~\ref{extendindep}, the number
of nonzero rows in the reduced matrix is the dimension of $S$ and these
rows form a basis for $S$.  So:
\[
\left(\begin{array}{rrrr} 1 & 0 & -1 & 0 \\ 0 & 1 & 1 & 1 \\ 5
& 4 & -1 & 4 \end{array}\right) \longrightarrow \left(\begin{array}
{rrrr} 1& 0 & -1 & 0 \\ 0 & 1 & 1 & 1 \\ 0 & 0 & 0 & 0
\end{array}\right).
\]

\end{solution}
\end{exercise}


%%%%%%%%%%%%%%%%%%%%%%%%%%%%%%%%%%%%%%%%%%%%%%%%%%%%%%%%%%%%%%%%


\problemlabel

\exerciselabel{4}{5.5}\begin{exercise} \label{c5.5.3}
Find a basis for the null space of
\[
A =\left(\begin{array}{rrrr} 1 & 0 & -1 & 2\\ 1 & -1 & 0 & 0\\
4 & -5 & 1 & -2 \end{array} \right).
\]
What is the dimension of the null space of $A$?

\begin{solution}

\ans The vectors $(1,1,1,0)$ and $(-2,-2,0,1)$ form a basis for the
nullspace of $A$; therefore the dimension of the nullspace is $2$.

\soln Find the set of solutions to $Ax = 0$ by solving
\[
\left(\begin{array}{rrrr} 1 & 0 & -1 & 2 \\ 1 & -1 & 0 & 0 \\ 4
& -5 & 1 & -2 \end{array}\right) \left(\begin{array}{r} x_1 \\ x_2
\\ x_3 \\ x_4 \end{array}\right) = 0.
\]
To solve, row reduce $A$, obtaining
\[
\left(\begin{array}{rrrr} 1 & 0 & -1 & 2 \\ 0 & 1 & -1 & 2 \\ 0
& 0 & 0 & 0 \end{array}\right).
\]
So the set of solutions to $Ax = 0$ can be written
\[
\left(\begin{array}{r} x_1 \\ x_2 \\ x_3 \\ x_4
\end{array}\right) = \left(\begin{array}{c} x_3 - 2x_4 \\ x_3 - 2x_4
\\ x_3 \\ x_4 \end{array}\right) = x_3\left(\begin{array}{r} 1 \\ 1
\\ 1 \\ 0 \end{array}\right) + x_4\left(\begin{array}{r} -2 \\ -2
\\ 0 \\ 1 \end{array}\right).
\]

\end{solution}
\end{exercise}


%%%%%%%%%%%%%%%%%%%%%%%%%%%%%%%%%%%%%%%%%%%%%%%%%%%%%%%%%%%%%%%%


\problemlabel

% 5.5 Dimension and Bases 


\exerciselabel{10}{5.5}\begin{exercise}\label{mc.exercise14}

Determine whether each of the following statements is true or false and explain your answer.  
\begin{enumerate}%[label=(\alph*)]
\item If $A$ is an $m\times n$ matrix and the equation $AX=b$ is consistent for some $b$, then the columns of $A$ span $\mathbb{R}^m$.
\item Let $A$ and $B$ be $n\times n$ matrices. If $AB = BA$ and if $A$ is invertible, then $A^{-1} B= B A^{-1}$. 
\item If $A$ and $B$ are $m\times$n matrices, then both $AB^t$ and $A^t B$ are defined.
\item Similar matrices always have the same eigenvectors.
\item If $u,v,w$ are vectors such that $\{u,v\},$ $\{u,w\},$ and $\{v,w\}$ are linearly independent sets, then $\{u,v,w\}$ is a linearly independent set.
\item Let $\{v_1,v_2,v_3\}$ be a basis for a vector space $V$. If $U$ is a subspace of $V,$ then some subset of $\{v_1,v_2,v_3\}$ is a basis for $U$.
\end{enumerate}
\begin{solution}

\ans 
We strike out those statements that are false and circle those that are true.
\begin{enumerate}%[label=(\alph*)]
\item {If $A$ is an $m\times n$ matrix and the equation $AX=b$ is consistent for some $b$, then the columns of $A$ span $\mathbb{R}^m$.}
\item {Let $A$ and $B$ be $n\times n$ matrices. If $AB = BA$ and if $A$ is invertible, then $A^{-1} B= B A^{-1}$.}
\item {If $A$ and $B$ are $m\times$n matrices, then both $AB^t$ and $A^t B$ are defined.}
\item {Similar matrices always have the same eigenvectors.}
\item {If $u,v,w$ are vectors such that $\{u,v\},$ $\{u,w\},$ and $\{v,w\}$ are linearly independent sets, then $\{u,v,w\}$ is a linearly independent set.}
\item {Let $\{v_1,v_2,v_3\}$ be a basis for a vector space $V$. If $U$ is a subspace of $V,$ then some subset of $\{v_1,v_2,v_3\}$ is a basis for $U$.}
\end{enumerate}

\soln \begin{enumerate}%[label=(\alph*)]
\item This is false. For example, let $A=\Matrix{1 & 0 \\ 0 & 0}$ and $b=\Matrix{1\\0}$. Then the system of equations $AX=b$ is consistent with solution $X=\Matrix{1\\0}$. 

\item This is false. For example, let $A=\Matrix{1 & 1\\ 0 & 1}$ and $B=\Matrix{1 & 0\\ 0 & 2}$. Then $A^{-1}=\Matrix{1 & -1\\ 0 & 1}$ and 
\[
A^{-1}B = \Matrix{1 & -2\\ 0 & 2} \neq \Matrix{1 & -1\\ 0 & 2} = BA^{-1}.
\]

\item This is true.  The transpose of an $m\times n$ matrix is an $n\times m$ matrix. Therefore 
\[
AB^t \mbox{ is }  (m\times n)(n\times m) = m\times m
\]
and
\[
A^tB \mbox{ is }  (n\times m)(m\times n) = n\times n.
\]

\item This is false. Suppose $A$ and $B$ are similar matrices. Then, there exists an invertible matrix $P$ such that $A=P^{-1}BP$. Let $B=\Matrix{ 0 & 1 \\ 1 & 0}$ and $P=\Matrix{ 1 & 1 \\ 0 & 1}$. Then $P^{-1}=\Matrix{ 1 & -1 \\ 0 & 1}$ and $A=P^{-1}BP=\Matrix{ -1 & 0 \\ 1 & 1}$. $B$ has eigenvectors $(1,1)$ and $(1,-1)$. $A$ has eigenvectors $(0,1)$ and $(2,-1)$. These are not the same. 

\item This is false. For example, let $V=\R^3$, $u=e_1$, $v=e_2$ and $w=e_1+e_2$. Then $\{u,v,w\}$ is a linearly dependent set, but $\{u,v\},$ $\{u,w\},$ and $\{v,w\}$ are all linearly independent sets.

\item This is false. For example, let $V=\R^2$ and let $v_1=e_1$ and $v_2=e_2$ be the standard basis vectors. Let $U$ be the one-dimensional subspace with basis $\{e_1+e_2\}$. Then no subset of $\{e_1,e_2\}$ is a basis for $U$.
\end{enumerate}
\end{solution}
\end{exercise}


%%%%%%%%%%%%%%%%%%%%%%%%%%%%%%%%%%%%%%%%%%%%%%%%%%%%%%%%%%%%%%%%




\end{document}
